Concurrent Stochastic Games (CSGs) \cite{Kwiatkowska2020,Kwiatkowska2021,KNPS19,KNPS22} have been implemented in various scenarios \cite{BAOUYA2024101161}, as evidenced by the literature review in the PRISM library  \cite{prismusecase}. This research leverages a variation of the stochastic game model presented in \cite{Javier2015,Javier201152} to identify optimal adaptation strategies through collaborative human maneuvers. Notably, the work in \cite{Javier201152} incorporates the human factor as a state of availability (not as a player) within the model.  In contrast, the research presented in \cite{Ray2023,RayBanerjee2023}, focuses on Multi-access Edge Computing (MEC) and proposes a service placement policy that utilizes both static (prioritized placement) and dynamic (runtime adjustments) strategies to optimize latency, resource usage, and energy consumption.


Several relevant research papers have investigated the maintenance of satellite systems \cite{Hoque2015,Yu2015, Zhaoguang2013}. In \cite{Zhaoguang2013}, the authors propose modeling a satellite system using Continuous-Time Markov Chains (CTMCs). This approach allows them to portray the impact of various factors on satellite reliability, including failures related to solar radiation and maintenance. The authors in \cite{Hoque2015}  build upon the model presented in \cite{Zhaoguang2013} by incorporating Erlang distributions instead of the exponential distributions supported by CTMCs. This change leads to more accurate results when comparing qualitative findings. Finally, authors in \cite{Yu2015} model the system using Markov Decision Processes (MDPs) to account for communication between the satellite system and ground stations. Additionally, they utilize the \emath{\pi-calculus} to model the system's semantics. Building on the findings of \cite{Zhaoguang2013}, the authors in \cite{Zhaoguang2016} model the reliability of a satellite constellation using CTMCs. \cmt{However, the impact of human interaction on maintenance costs is not addressed in any of the contributions as a game model.}