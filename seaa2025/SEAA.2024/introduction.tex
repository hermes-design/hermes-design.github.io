
Satellite systems have become a crucial part of our daily lives. The demand for reliable communication anywhere on Earth has driven innovation in the space industry, fostering competition among new space entrepreneurs like SpaceX and Amazon. These satellite constellations serve a wide range of purposes, including remote sensing for applications like the Internet of Things (IoT) and weather forecasting, as well as supporting critical military operations.  Therefore, Reliability, Availability, and Maintainability (RAM) are paramount considerations during the design phase.  A focus on RAM ensures systems are dependable and maintainable, minimizing the costs and complexities associated with potential repairs.

Formal verification \cite{Kwiatkowskaprism2011} is a powerful technique for ensuring the correctness and reliability of complex systems. It utilizes various formalisms, each suited to specific use cases. Some common formalisms include Markov Decision Processes (MDPs), Continuous-Time Markov Chains (CTMCs), and Concurrent Stochastic Games (CSGs). Stochastic games verification \cite{Kwiatkowska2019} allows for the generation of quantitative correctness assertions about a system's behavior (e.g. \quot{The object recognition system can correctly identify pedestrians with a probability of at least 95\%, even in challenging lighting conditions}.), where the required behavioral properties are expressed in quantitative extensions of temporal logic. The problem of strategy synthesis constructs an optimal strategy for a player, or coalition of players, to ensure a desired outcome (property) is achieved. The formalism of Concurrent stochastic multi-player games (CSGs) \cite{Kwiatkowska2019,Kwiatkowska2020} permits players to choose their actions concurrently in each state of the model. This approach captures the true essence of concurrent interaction, where agents make independent choices simultaneously without perfect knowledge of others' actions. However, although algorithms for verification and strategy synthesis of CSGs have been implemented in PRISM-games\cite{Kwiatkowska2021}, their adoption for RAM analysis has not been investigated.


This paper demonstrates how to accurately model satellite systems and verify their Reliability, Availability, and Maintainability (RAM) properties using the PRISM-games model checker. The PRISM-games tool extends the capabilities of classical probabilistic model checkers, which have been widely applied to verify the correctness and effectiveness of hardware and software designs \cite{prismmodelchecker}. However, classical models in probabilistic automata (PA) often focus on a single perspective.  Our approach leverages Concurrent Stochastic Games (CSGs) to capture the collaborative maintenance and repair of satellite systems \cite{Hoque2015,Yu2015,Zhaoguang2013}. This allows us to model multiple players involved in the system's operation, including the environment (which introduces failures and planned/unplanned interruptions), on-orbit maintenance maneuvers (responsible for software update moving to a redundant satellite), and ground control (responsible for satellite preparation and launch). CSGs are particularly well-suited for this task as they comprehensively represent failure management and collaborative maintenance.


\paragraph*{Outline}The remainder of this paper is structured as follows. Section~\ref{sec:rw} reviews related work. Section~\ref{Preliminaries} provides the necessary background on Concurrent Stochastic Games (CSGs) and the PRISM-games language. Section~\ref{sattelitemodel} presents our proposed modeling approach for satellite systems. We then perform a quantitative analysis of the satellite system's behavior under failure scenarios in Section~\ref{sec:useCase}. Finally, Section~\ref{conclusion} concludes the paper and suggests directions for future research.