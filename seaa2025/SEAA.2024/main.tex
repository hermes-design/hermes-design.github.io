% This is samplepaper.tex, a sample chapter demonstrating the
% LLNCS macro package for Springer Computer Science proceedings;
% Version 2.21 of 2022/01/12
%
\documentclass[conference]{IEEEtran}
\IEEEoverridecommandlockouts
%
\usepackage{changepage}
\usepackage{mathtools}
\usepackage{graphicx}
\usepackage{xcolor}
\usepackage{tcolorbox}
\usepackage{makeidx}
    \usepackage{float}
    \usepackage{graphicx}
\usepackage{makeidx}  % allows for indexgeneration
\usepackage[english]{babel} % un troisième package
\usepackage{amssymb}
\usepackage{syntax}
\usepackage{multirow}
\usepackage{array}
%\usepackage[table]{xcolor}
\usepackage{colortbl}
\usepackage{enumitem}
\usepackage{booktabs}
\usepackage{listings}
%\usepackage{verbatim}
\usepackage{caption}
\usepackage{subcaption}
\usepackage{courier}
\usepackage{amsmath}
\usepackage{mathtools}
\usepackage{wasysym}
%\usepackage[sort=appearance]{spbasic}
%\usepackage{mathrsfs}
%\usepackage{sectsty}
%\usepackage{mathrsfs}
%\usepackage[numbers]{natbib}
%\usepackage[numbers,sort]{natbib}
%\usepackage[numbers]{natbib}
%\usepackage[backend=biber,style=numeric,sorting=none]{biblat‌ex}
%\bibliographystyle{unsrt}
%\setcitestyle{authoryear,open={(},close={)}}
%\setcitestyle{square,aysep={},yysep={;}}
%\usepackage[sort&compress]{natbib}
%\usepackage{amsthm}
%\usepackage{mathrsfs}
%\usepackage[authoryear]{natbib}
%\usepackage{amsthm}
\usepackage{tikz}
\usepackage[linesnumbered]{algorithm2e}
%\usepackage{algorithm}
%\usepackage{algorithmic}
\usepackage{algpseudocode}
\definecolor{bluekeywords}{rgb}{0.13, 0.13, 1}
\definecolor{greencomments}{rgb}{0, 0.5, 0}
\definecolor{redstrings}{rgb}{0.9, 0, 0}
\definecolor{graynumbers}{rgb}{0.5, 0.5, 0.5}
\definecolor{lightgray}{rgb}{0.83, 0.83, 0.83}
\definecolor{magnolia}{rgb}{0.93, 1.96, 1.3}


%\newtcolorbox{units}{before=\par\smallskip\centering,after=\par,hbox}

\definecolor{commentgreen}{RGB}{2,112,10}
\definecolor{eminence}{RGB}{108,48,130}
\definecolor{weborange}{RGB}{255,165,0}
\definecolor{frenchplum}{RGB}{129,20,83}
\definecolor{teagreen}{rgb}{0.82, 0.94, 0.75}
\definecolor{asparagus}{rgb}{0.53, 0.66, 0.42}
\usepackage[hyphens]{url}
\usepackage{hyperref}
\hypersetup{
  colorlinks,
  citecolor= 	blue,
  linkcolor= 	blue,
  urlcolor= 	blue}
    \usepackage{breakurl}
    \hypersetup{colorlinks=true,breaklinks=true}
  \usepackage{eucal}
  \usepackage{tabularx}
\usepackage{rotating}
\usepackage[T1]{fontenc}
\usepackage[utf8]{inputenc}
%\usepackage{amsmath,amsfonts}

\usepackage[bitstream-charter]{mathdesign}
\usepackage[T1]{fontenc}
\usepackage{booktabs}
%\normalfont
\usepackage{listings}     
\usepackage{lstautogobble}  % Fix relative indenting
\usepackage{color}          % Code coloring
\usepackage{zi4}            % Nice font

\usepackage{lineno}



%\linenumbers


\DeclareCaptionFont{white}{\color{white}} 
\DeclareCaptionFormat{listing}{\colorbox{gray}{\parbox{\dimexpr\linewidth-1.9\fboxsep\relax}{#1#2#3}}}

\captionsetup[lstlisting]{format=listing,labelfont=white,textfont=white} 

\newcommand{\gtext}[1] {\textcolor{forestgreen}{#1}}


\newcommand{\keyw}[1] {\texttt{\textbf{#1}}}

\newcommand{\ie}{i.e., }
\definecolor{forestgreen}{rgb}{0.0, 0.5, 0.0}

\definecolor{aliceblue}{rgb}{0.94, 0.97, 1.0}



\newcommand{\eclipse}[1] {\textcolor{eminence}{\texttt{\textbf{#1}}}}

\newcommand{\num}[1] {\texttt{#1}}

\newcommand{\key}[1] {\texttt{\textbf{#1}}}


\newcommand{\sset}[1] {\llbracket #1 \rrbracket}

\newcommand{\eg}{e.g. }

\newcommand{\ai}{ Artificial Intelligence}

\newcommand{\ecircle}[1] {\textcircled{{\small #1}}}

\newcommand{\cmt}[1] {\textcolor{blue}{#1}}


\newcommand{\tab}[1]{Table \ref{#1}}

\newcommand{\algo}[1]{Algorithm \ref{#1}}

\newcommand{\commenting}[1] {\textcolor{blue}{#1}}

\newcommand{\quot}[1] {``#1''}

\newcommand{\emath}[1] {$#1$}

\newcommand{\emathtt}[1] {$\mathtt{#1}$}

\newcommand{\msym}[1] { $\mathscr{#1}$ }

\newcommand{\alg}[1] {Algorithm \ref{#1}}

\newcommand{\lst}[1] {Listing \ref{#1}}

\newcommand{\link}[2] {\texttt{#1}$\rightarrow$ \texttt{#2}}

\newtheorem{mydef}{\normalfont \textbf{Definition}}

\newcommand{\AB}[1] {\textcolor{red}{#1}}

\newcommand{\BH}[1] {\textcolor{blue}{#1}}

\newcommand{\gl}[1] {\langle {#1} \rangle}
\newcommand{\effect} {\textrm{Effect}}
\newcommand{\push} {\textrm{Push}}
\newcommand{\pull} {\textrm{Pull}}

\newcommand\myeq{\stackrel{\mathclap{\footnotesize\mbox{def}}}{=}}

% draw a frame around given text
\newcommand{\framedtext}[1]{%
\par%
\noindent\fbox{%
    \parbox{\dimexpr\linewidth-2\fboxsep-2\fboxrule}{#1}%
}%
}

\definecolor{commentgreen}{RGB}{2,112,10}
\definecolor{eminence}{RGB}{108,48,130}
\definecolor{weborange}{RGB}{255,165,0}
\definecolor{frenchplum}{RGB}{129,20,83}

\newtheorem{example}{\normalfont \textbf{Example}}  % Create the "Example" environment with plain style

\newcommand{\code}[1] {\texttt{#1}}

%\spnewtheorem{myexample}{Example}[section]{\bfseries}{\itshape}


\SetSymbolFont{operators}   {normal}{OT1}{cmr} {m}{n}
\SetSymbolFont{letters}     {normal}{OML}{cmm} {m}{it}
\SetSymbolFont{symbols}     {normal}{OMS}{cmsy}{m}{n}
\SetSymbolFont{largesymbols}{normal}{OMX}{cmex}{m}{n}
\SetSymbolFont{operators}   {bold}  {OT1}{cmr} {bx}{n}
\SetSymbolFont{letters}     {bold}  {OML}{cmm} {b}{it}
\SetSymbolFont{symbols}     {bold}  {OMS}{cmsy}{b}{n}
\SetSymbolFont{largesymbols}{bold}  {OMX}{cmex}{m}{n}

\SetMathAlphabet{\mathbf}{normal}{OT1}{cmr}{bx}{n}
\SetMathAlphabet{\mathsf}{normal}{OT1}{cmss}{m}{n}
\SetMathAlphabet{\mathit}{normal}{OT1}{cmr}{m}{it}
\SetMathAlphabet{\mathtt}{normal}{OT1}{cmtt}{m}{n}
\SetMathAlphabet{\mathbf}{bold}  {OT1}{cmr}{bx}{n}
\SetMathAlphabet{\mathsf}{bold}  {OT1}{cmss}{bx}{n}
\SetMathAlphabet{\mathit}{bold}  {OT1}{cmr}{bx}{it}
\SetMathAlphabet{\mathtt}{bold}  {OT1}{cmtt}{m}{n}


%\renewenvironment{myexample}[1][]{%
% \setcounter{myexample}{1}
% \par\vspace{5pt}\noindent
% \fbox{\textbf{Example~\thesection.\arabic{myexample}}}
% \hrulefill\par\vspace{10pt}\noindent\rmfamily}
% {\par\noindent\hrulefill\vrule width10pt height2pt depth2pt\par}

%\makeatletter
%\@addtoreset{myexample}{section}
%\makeatother
\usepackage{tikz}
\usetikzlibrary{calc}
\usepackage{tcolorbox}

%%%%%%%%%%%%%%%%%%%%%%%%%%%%%%%%%%%%%%
\usetikzlibrary{calc,shadows.blur}
\tcbuselibrary{skins} % Assuming you have the "skins" library installed

\newtcolorbox{resp}[2][]{%
  enhanced jigsaw,
  colback=white,%
  colframe=gray,
  size=small, % Choose which "size=small" to keep
  boxrule=1pt,
  title=#2,
  halign title=flush center,
  coltitle=black,
  drop shadow=black!3!white,
  attach boxed title to top left={xshift=1cm,yshift=-\tcboxedtitleheight/2,yshifttext=-\tcboxedtitleheight/2},
  minipage boxed title=3cm,
  boxed title style={%
    colback=white,
    size=fbox, % Choose which "size=fbox" to keep
    boxrule=1pt,
    boxsep=2pt,
    underlay={%
      \coordinate (dotA) at ($(interior.west) + (-0.5pt,0)$);
      \coordinate (dotB) at ($(interior.east) + (0.5pt,0)$);
      \begin{scope}
        \clip (interior.north west) rectangle ([xshift=3ex]interior.east);
        \filldraw [white, blur shadow={shadow opacity=60, shadow yshift=-.75ex}, rounded corners=2pt] (interior.north west) rectangle (interior.south east);
      \end{scope}
      \begin{scope}[gray!80!black]
        \fill (dotA) circle (2pt);
        \fill (dotB) circle (2pt);
      \end{scope}
    },
  },
  #1, % Move "Learned" to before the closing curly brace
}%

\newtcolorbox{boxD}{
    colback = white, 
    colframe = black, 
    boxrule = 0pt, 
    toprule = 3pt, % top rule weight
    bottomrule = 3pt % bottom rule weight
}
\newtcolorbox{boxF}{
    colback = yellow!5!white,
    enhanced,
    boxrule = 1.5pt, 
    colframe = white, % making the base for dash line
    borderline = {1.1pt}{0pt}{main, dashed} % add "dashed" for dashed line
}

\newtcolorbox{boxC}{
    colback = blue!0!white,  % background color
    boxrule = 0pt  % no borders
}
\newcommand{\fig}[1]{Figure \ref{#1}}
%%%%%%%%%%%%%%%%%%%%%%%%%%%%%%%%%
% Exercise Environment
%\newenvironment{exercise}{\exerciseinner\mbox{}\par\bigskip}{\endexerciseinner}
%\newcommand{\theexercise}{\theexerciseinner}

% Exercise Environment
%\tcolorboxenvironment{exercise}{
%  breakable,
%   enhanced,
%   colback=gray!7!white,
%   parbox=false, drop fuzzy shadow
%}

%%%%%%%%%%%%%%%%%%%%%%%%%%
\DeclareCaptionFont{white}{\color{white}}
\DeclareCaptionFormat{listing}{%
    \colorbox{black}{\parbox{\dimexpr\textwidth-2\fboxsep}{\textbf{\textcolor{white}{#1#2#3}}}}}
\captionsetup[lstlisting]{format=listing,labelfont=white,textfont=white}

\SetSymbolFont{operators}   {bold}{OT1}{cmr} {m}{n}
\SetSymbolFont{letters}     {bold}{OML}{cmm} {m}{it}
\SetSymbolFont{symbols}     {bold}{OMS}{cmsy}{m}{n}
\SetSymbolFont{largesymbols}{bold}{OMX}{cmex}{m}{n}
\SetSymbolFont{operators}   {bold}  {OT1}{cmr} {bx}{n}
\SetSymbolFont{letters}     {bold}  {OML}{cmm} {b}{it}
\SetSymbolFont{symbols}     {bold}  {OMS}{cmsy}{b}{n}
\SetSymbolFont{largesymbols}{bold}  {OMX}{cmex}{m}{n}

\SetMathAlphabet{\mathbf}{bold}{OT1}{cmr}{bx}{n}
\SetMathAlphabet{\mathsf}{bold}{OT1}{cmss}{m}{n}
\SetMathAlphabet{\mathit}{bold}{OT1}{cmr}{m}{it}
\SetMathAlphabet{\mathtt}{bold}{OT1}{cmtt}{m}{n}
\SetMathAlphabet{\mathbf}{bold}  {OT1}{cmr}{bx}{n}
\SetMathAlphabet{\mathsf}{bold}  {OT1}{cmss}{bx}{n}
\SetMathAlphabet{\mathit}{bold}  {OT1}{cmr}{bx}{it}
\SetMathAlphabet{\mathtt}{bold}  {OT1}{cmtt}{m}{n}

\def\llbracket{[\![}
\def\rrbracket{]\!]}

\newcommand{\CO}[1] {\langle\langle #1 \rangle\rangle}



\newcommand\setItemnumber[1]{\setcounter{enumi}{\numexpr#1-1\relax}}

\newcommand{\gparrow}[1] {\lhook\joinrel\xrightarrow{#1} }

\newcommand{\project} {  \url{https://hermes-design.github.io/ieaaie24.html} }


\newcommand{\acisiot} {s\textbf{A}fety and se\textbf{C}ur\textbf{I}ty as\textbf{S}urrance for critical \textbf{IoT} systems}

\usepackage{lineno}

\usepackage{calrsfs}
\DeclareMathAlphabet{\pazocal}{OMS}{zplm}{m}{n}
\newcommand{\La}{\mathcal{L}}
\newcommand{\Lb}{\pazocal{L}}
\newenvironment{BoldDef}[1]{\normalfont \textbf{Definition.} #1}{\par}

\begin{document}
\counterwithin{lstlisting}{section}


\counterwithin{lstlisting}{section}

%Assessing Security Risks on Edge Servers Implementing RabbitMQ Protocol through Concurrent Stochastic Games
\title{Model-Based Reliability, Availability, and Maintainability Analysis for Satellite Systems with Collaborative Maneuvers via Stochastic Games}



%Reliability, Availability and Maintainability Analysis for Satellite Systems Based on Concurrent Stochastic Games and Collaborative manoeuvers

\author{Abdelhakim Baouya$^{1}$, Brahim Hamid$^{1}$, Otmane Ait Mohamed$^{2}$, Saddek Bensalem$^{3}$\\
	\normalsize $^{1}$University of Toulouse, IRIT, France\\
	\normalsize abdelhakim.baouya@irit.fr, brahim.hamid@irit.fr\\
 	\normalsize $^{2}$Concordia University, CANADA\\
	\normalsize otmane.aitmohamed@concordia.ca\\
  	\normalsize $^{3}$VERIMAG, Université Grenoble Alpes, Grenoble, France\\
	\normalsize saddek.bensalem@univ-grenoble-alpes.fr\\
}


%, \emph{ACM Member}

\maketitle

\begin{abstract}
Space-based navigation systems (GPS, GLONASS) rely on satellites to operate in orbit and have lifetimes of 10 years or more. Engineers employ Reliability, Availability, and Maintainability (RAM) analysis during the design phase to maximize a satellite's mean time between failures (MTBF). These design parameters help to optimize maintenance plans, enhance overall reliability, and extend the satellite's lifespan. The paper presents a novel approach using concurrent stochastic games (CSG) to model a single satellite with logical and formal specifications of RAM properties in rPATL. We leverage the PRISM-games model checker for quantitative analysis while considering collaborative behaviors between involved players in orbit and on the ground. This CSG-based approach offers a rich design space where actors considered as players involved in satellite maintenance can collaborate and learn optimal strategies.

\end{abstract}

\begin{IEEEkeywords}
Navigation Satellite Systems, Reliability, Availability, Maintainability, Concurrent Stochastic Games
\end{IEEEkeywords}

\section{Introduction}

\begin{sloppypar}

Satellite systems have become a crucial part of our daily lives. The demand for reliable communication anywhere on Earth has driven innovation in the space industry, fostering competition among new space entrepreneurs like SpaceX and Amazon. These satellite constellations serve a wide range of purposes, including remote sensing for applications like the Internet of Things (IoT) and weather forecasting, as well as supporting critical military operations.  Therefore, Reliability, Availability, and Maintainability (RAM) are paramount considerations during the design phase.  A focus on RAM ensures systems are dependable and maintainable, minimizing the costs and complexities associated with potential repairs.

Formal verification \cite{Kwiatkowskaprism2011} is a powerful technique for ensuring the correctness and reliability of complex systems. It utilizes various formalisms, each suited to specific use cases. Some common formalisms include Markov Decision Processes (MDPs), Continuous-Time Markov Chains (CTMCs), and Concurrent Stochastic Games (CSGs). Stochastic games verification \cite{Kwiatkowska2019} allows for the generation of quantitative correctness assertions about a system's behavior (e.g. \quot{The object recognition system can correctly identify pedestrians with a probability of at least 95\%, even in challenging lighting conditions}.), where the required behavioral properties are expressed in quantitative extensions of temporal logic. The problem of strategy synthesis constructs an optimal strategy for a player, or coalition of players, to ensure a desired outcome (property) is achieved. The formalism of Concurrent stochastic multi-player games (CSGs) \cite{Kwiatkowska2019,Kwiatkowska2020} permits players to choose their actions concurrently in each state of the model. This approach captures the true essence of concurrent interaction, where agents make independent choices simultaneously without perfect knowledge of others' actions. However, although algorithms for verification and strategy synthesis of CSGs have been implemented in PRISM-games\cite{Kwiatkowska2021}, their adoption for RAM analysis has not been investigated.


This paper demonstrates how to accurately model satellite systems and verify their Reliability, Availability, and Maintainability (RAM) properties using the PRISM-games model checker. The PRISM-games tool extends the capabilities of classical probabilistic model checkers, which have been widely applied to verify the correctness and effectiveness of hardware and software designs \cite{prismmodelchecker}. However, classical models in probabilistic automata (PA) often focus on a single perspective.  Our approach leverages Concurrent Stochastic Games (CSGs) to capture the collaborative maintenance and repair of satellite systems \cite{Hoque2015,Yu2015,Zhaoguang2013}. This allows us to model multiple players involved in the system's operation, including the environment (which introduces failures and planned/unplanned interruptions), on-orbit maintenance maneuvers (responsible for software update moving to a redundant satellite), and ground control (responsible for satellite preparation and launch). CSGs are particularly well-suited for this task as they comprehensively represent failure management and collaborative maintenance.


\paragraph*{Outline}The remainder of this paper is structured as follows. Section~\ref{sec:rw} reviews related work. Section~\ref{Preliminaries} provides the necessary background on Concurrent Stochastic Games (CSGs) and the PRISM-games language. Section~\ref{sattelitemodel} presents our proposed modeling approach for satellite systems. We then perform a quantitative analysis of the satellite system's behavior under failure scenarios in Section~\ref{sec:useCase}. Finally, Section~\ref{conclusion} concludes the paper and suggests directions for future research.
\end{sloppypar}

\section{ Related work}
\label{sec:rw}
\begin{sloppypar}
Several works in the literature have addressed the \cmt{quality and deployment} of satellite systems. In \cite{Zhaoguang2013}, the authors propose \cmt{to use CTMC to model} satellite systems, demonstrating how this approach can analyze the impact of \cmt{different} factors, \cmt{such as} solar radiation, on satellite reliability and maintenance. Building upon this work, \cite{Hoque2015} incorporates Erlang distributions into the CTMC framework to further refine the modeling of maintenance and scrubbing activities. Extending these studies, \cite{Zhaoguang2016} focuses on modeling the reliability of an entire satellite constellation using CTMCs. \cmt{Additionally}, \cite{Baouyaseaa2024} \cmt{broadness} the scope by incorporating human \cmt{capabilities} in maintaining satellite systems within a \cmt{game-theoritical} model.

While \cmt{significant} research has focused on the quality and deployment of US GPS satellites, research activities \cmt{explicitely} addressing the reliability and performance of the SAR/Galileo satellite system are less prevalent in the literature. \cmt{Known studies addressing this topic} include \cite{Alegre2014,Inone2027,Lewandowski2008}. In \cite{Lewandowski2008}, the authors investigate a hybrid communication infrastructure combining terrestrial networks with enhanced Galileo SAR for firefighter missions. Using simulations, they compare the advanced Galileo SAR system with the existing Cospas-Sarsat system. They identify performance metrics and determine the optimal number of Galileo satellites to enhance SAR service quality and minimize emergency response times.  The authors in \cite{9115548} evaluate the performance of four Galileo satellites launched in 2018 during their first year of operational service ( to reach 22 satellites). Key performance indicators, including signal-in-space status, availability, and ranging accuracy, were analyzed. Results demonstrate high signal health and availability, with minimal signal interruptions. Moreover, the numerical results demonstrate high reliability for the most recent additions to the Galileo satellite constellation. The authors in \cite{Inone2027} focus on the development of the SAR/Galileo ground system in Korea, outlining a roadmap strategy for the next-generation search and rescue system with the goal of enhancing rescue efficiency by improving location accuracy and reducing response times. The authors in \cite{Alegre2014} investigate methods to improve the availability of the Return Link Service (RLS) in the Galileo SAR system. \cmt{Their} proposed solutions, \cmt{utilize} Network Coding, \cmt{to improve} RLM reception by mitigating signal losses due to harsh weather or \cmt{obstacles and} considering backward compatibility and system complexity. \cmt{However, none} of the reviewed research explicitly addresses the quality of Galileo SAR services from the perspective of reliable intervention, \cmt{especially regarding} the dependability parameters \cmt{mentioned} in the documentation and the psychological status of the person in distress.
\end{sloppypar}

\section{Background on Concurrent Stochastic  Games}
\label{Preliminaries}
\begin{sloppypar}
Probabilistic Model Checking using PRISM \cite{Kwiatkowskaprism2011} relies on constructing a formal model, typically represented using appropriate storage structures. The verification process is then performed by applying a suite of algorithms implemented within the PRISM engine \cite{engines}. For our analysis, we employ Continuous-Time Markov Chains (CTMCs) \cite{Kwiatkowska2007}, a well-established modeling technique for evaluating reliability and performance. The CTMC involves a set of states and a transition matrix \emath{\textbf{R}: S \times S \rightarrow \mathbb{R}_{\geq 0}}.  The rate specifies the delay before a transition between states s and s' takes place \emath{\textbf{R}(s,s')}, where the probability between \emath{s} and \emath{s'} take within time t is given by the value \emath{1-e^{-\textbf{R}(s,s') \times t}}. Based on research using PRISM for CTMC modeling, as outlined in \cite{prismctmc}, exponentially distributed delays are often considered suitable for modeling electronic component lifetimes and inter-arrival times. 

The PRISM model is composed of a set of modules that can synchronize. Each module is characterized by variables and commands (or transitions). The valuations of these variables represent the state of the module. The behavior of each module is described using a set of commands, each of which follows the following format:
\[
[a] \ g \ \rightarrow \ \lambda: u
\]
This indicates that if the guard condition \( g \) evaluates to true, then the update \( u \) is enabled to occur with a rate of \( \lambda \) for action \( a \). A guard is a boolean formula constructed from the module variables. The update \( u \) is an evaluation of variables expressed as a conjunction of assignments: \emath{v_{i}' = val_{i} + \ldots + v_{n}' = val_{n}} where \( v_{i} \in V \), with \( V \) being a set of local and global variables, and \( val_{i} \) are values evaluated via expressions denoted by \( \theta \) such that \( \theta: V \rightarrow \mathbb{D} \), where \( \mathbb{D} \) is the domain of the variables.

Two types of reward functions are highlighted. The action reward function assigns a real value to each state-action. This value is accumulated when the action \( a \) is selected in the state \( s \). Additionally, the state reward function, denoted as \( r_{S} : S \longrightarrow \mathbb{R} \), assigns a real value to each state \( s \). This value is accumulated when the state \( s \) is reached.

Properties are typically expressed in Continuous Stochastic Logic (CSL) \cite{kwiatkowska2002approximate}, a stochastic variant of the well-known Computational Tree Logic (CTL).  For instance, the following property expressed in natural language: \emph{Is the probability of that eventually the system failure occurring within 100 time units is less than 0.001} is expressed  as: \emath{ P_{<0.001} [ F^{\leq 100} \ fail ]}
Here, \(fail\) is the label that refers to the system failure states. 
Regarding the reward structure, the property expressed in natural language: \emph{What is the amount of reward accumulated over a specific 100 times units ?} is expressed in CSL as:
\emath{ R\{"up"\}=? [C \leq 100]}.


\end{sloppypar}

\section{Formal modeling of satellite systems}
\label{sattelitemodel}
\begin{sloppypar}
PRISM-games is a probabilistic model checker designed specifically to analyze Concurrent Stochastic Games (CSGs), which involve multiple players. The PRISM games supports verifying properties expressed in a logic like rPATL (an extension of PCTL), allowing reasoning about probabilities and rewards within the model.  This enables the creation of abstract state-based system models, like the one for a single satellite system illustrated in \fig{fig:usecase}.


\begin{figure*}[htbp]
    \centering
    		\includegraphics[width=500pt, height =180pt]{Dessin1.pdf}
    \caption{Satellite Maintenance Process Model \cite{Hoque2015,Yu2015}}
    \label{fig:usecase}
\end{figure*} 

\subsection{The system model}
This paper leverages a previously established satellite model described in \cite{Hoque2015,Yu2015,Zhaoguang2013}. The model considers the system's vulnerability to both scheduled and unscheduled interruptions throughout its lifecycle. Scheduled interruptions occur due to maintenance or software updates, leading to temporary signal unavailability for a fixed duration of \emath{t_{\alpha}=}4,320 hours. Unscheduled interruptions, such as those caused by solar radiation, can induce a Single Event Upset (SEU) in the satellite's signal. Unlike scheduled events, SEUs are unpredictable but also self-correcting, resolving automatically. Permanent failures, however, necessitate maneuvers in orbit or on the ground.


Upon satellite failure, ground and orbit control evaluate the best course of action. In some cases, the issue might be resolved remotely by sending software commands to the satellite. If the problem persists, deploying a redundant satellite from orbit as a replacement may be necessary. However, if no backup satellite is available, a new one will need to be manufactured and launched, introducing the risk of a launch failure.

The probability of a satellite failure is \emath{1-r}, where \emath{r} is its reliability calculated from the failure rate and Mean Time Between Failures (MTBF). Both the unscheduled and scheduled interruption times are \emath{t_{\alpha} = 4320 h}. If a failure occurs, there is an \emath{p_{\beta}=80\%} chance of resolving it on orbit by replacing the faulty satellite with a redundant one. If on-orbit repair is impossible, a new satellite needs to be built. The ground control team manages this process. \cmt{The times taken to decide to build a new satellite and for one to be manufactured are} \emath{t_{\gamma}=24}  hours and \emath{t_{\delta}=24}, respectively. Following a successful launch with \emath{p_{n} =90\%}, it takes another  \emath{t_{k}=24} hours for the new satellite to reach its operational position.


\subsection{On orbit and ground support capabilities}
\label{humanlabel}


The modeled system involves multiple state transitions driven by the \emph{environment}, \emph{on-orbit staff}, and \emph{ground staff}. Each group performs specific tasks:
\begin{itemize}
	\item \emph{Environment}: Triggers scheduled, unscheduled, and failure events (represented by the environment player, denoted as \emath{\mathcal{P}_{env}}) shown in red in \fig{fig:usecase}. Their actions are labeled as \emath{\alpha}.
	\item  \emph{On-Orbit Staff}: These personnel (represented by  \emath{\mathcal{P}_{o})} handle tasks like sending commands, updating software, and maneuvering the satellite into the position shown in blue in \fig{fig:usecase}. Their actions are labeled as \emath{\beta}.
	\item \emph{On-Ground Staff}:  The ground control team (represented by\emath{\mathcal{P}_{g}}) is responsible for monitoring satellite health, building new satellites when necessary, and performing launches (shown in green in \fig{fig:usecase}). Their actions are labeled as \emath{\omega}.
\end{itemize}


To model composability, we introduce a dedicated PRISM module that acts as a non-player. This module encapsulates the environment, on-orbit actions, on-ground actions, and the PRISM commands needed to synchronize their interactions. We define a CSG model, G, as a game modeling the parallel composition of the environment player \emath{\mathcal{P}_{env}}, on-orbit staff player \emath{\mathcal{P}_{o}}, and ground staff player \emath{\mathcal{P}_{g}}. This composition is coordinated by the non-player model \emath{\mathcal{P}_{\mathcal{R}}.}


\begin{mydef} \label{def:csg} \normalfont A concurrent stochastic game (CSG) for reasoning on Sattelite system maintenance is a tuple \emath{G =\gl{N, S, \bar{S}, A, \delta, AP, L}}:

\begin{itemize}
	\item \emath{N =\{\mathcal{P}_{env},\mathcal{P}_{o}, \mathcal{P}_{g}\}} is a finite set of players,

 	\item \emath{S=S_{env} \times S_{o} \times S_{g}} is a set of states, where \emath{S_{env}}, \emath{S_{o}}, and \emath{S_{g}} are states controlled by the system model, the environment model \emath{\mathcal{P}_{env}}, the on-orbit player model \emath{\mathcal{P}_{o}}, and on-ground model \emath{\mathcal{P}_{g}}, respectively \emath{\cmt{(S_{env} \cap  S_{o} \cap S_{g} =\emptyset)},}
    and \emath{\bar{S} \subseteq S } is a set of initial states,


	 \item \emath{A= A_{env} \times A_{o} \times A_{g} } where \emath{A_{env}},  \emath{A_{o},} and \emath{A_{g}} are the actions available to the environment model, on-orbit model, and the on-ground model, respectively,
    \item \emath{\delta : S \times A \longrightarrow Dist(S)} is a probabilistic transition function. Each player \cmt{\emath{\mathcal{P}_{env},\mathcal{P}_{o}, \mathcal{P}_{g}}} selects an action \emath{\alpha,\beta,\omega}, the state of the game is updated according to the distribution \emath{\delta(s, (\alpha,\beta,\omega)) \in Dist(S),} 

    \item \emath{AP} is a subset of all predicates that can be built over state variables. AP includes:
    \begin{itemize}
	\item \emath{goal}, achieved when a successful operation is reached.
     \end{itemize}
    \item \emath{L: S \longrightarrow 2^{AP}} is a labeling function that assigns each state  \emath{s \in S}  to a set of atomic propositions (\emath{AP}).
\end{itemize}
\end{mydef}


Following the definition of the CSG players, The non-player commands of \emath{\mathcal{P}_{\mathcal{R}}} that record the strategy of the CSGs model are expressed through the following transition: \emath{s_m\gparrow{\alpha,\beta,\omega}s'_m}, where \emath{\alpha} represents the label of the environment commands, \emath{\beta} denotes the on-orbit command, and \emath{\omega} denotes the on-ground commands. The non-player is modeled by the operation semantics rules \ref{s1} and \ref{s2}. The \ref{s1} is achieved by composing the modules \emath{\mathcal{P}_{env}}, \ \emath{\mathcal{P}_{o}}, and \emath{\mathcal{P}_{g}.} The \emph{standby} action refers to the idle position of the player in the CSG model. In this composition, the probability of achieving on-orbit or on-grounds tasks is determined by \cmt{\emath{\prod_{i=1}^{|N|} \lambda_{i}} such that \emath{\lambda_{i} \in \mathbb{R}}}.


\begin{figure*}[th]
\begin{boxD}
%\framedtext{
	      \begin{equation}\label{s1} \frac{ \sset{\mathcal{P}_{env}}= s_{i}\gparrow{\alpha}_{\lambda_{1}}s'_{i}  \bigwedge_{j=0}^{|A_{o}|}  \sset{\mathcal{P}_{o}}= s_{j} \gparrow{\beta_{j}}_{\lambda_{2}}s'_{j}  \wedge
       \sset{\mathcal{P}_{g}}= s_{k}\gparrow{\omega}_{\lambda_{3}} s'_{k} 
       \wedge
         \sset{\mathcal{P}_{\mathcal{R}}}= s_{m}\gparrow{\alpha,\beta,\omega}s'_{m}
       } {  \langle s_{i},\ldots,s_{j},\ldots s_{k}, \ldots s_{m},\theta\rangle  \xrightarrow{\alpha,\beta,\omega}_{\lambda_{1} \cdot \lambda_{2} \cdot \lambda_{3}}\langle s'_{i},\ldots,s'_{j},\ldots,s'_{k},\ldots,s'_{m},\theta'\rangle } \tag{\emph{On-Orbit}} \end{equation} where \emath{\alpha = Fail \wedge \omega= standby}
 
	     \begin{equation}\label{s2} \frac{ \sset{\mathcal{P}_{env}}= s_{i}\gparrow{\alpha}_{\lambda_{1}}s'_{i} \wedge \sset{\mathcal{P}_{o}}= s_{j} \gparrow{\beta}_{\lambda_{2}}s'_{j} \bigwedge_{k=0}^{|A_{g}|} 
       \sset{\mathcal{P}_{g}}= s_{k}\gparrow{\omega_{k}}_{\lambda_{3}} s'_{k} 
       \wedge
         \sset{\mathcal{P}_{\mathcal{R}}}= s_{m}\gparrow{\alpha,\beta,\omega}s'_{m} 
       } {  \langle s_{i},\ldots,s_{j},\ldots s_{k}, \ldots s_{m},\theta\rangle  \xrightarrow{\alpha,\beta,\omega}_{\lambda_{1} \cdot \lambda_{2} \cdot \lambda_{3}}\langle s'_{i},\ldots,s'_{j},\ldots,s'_{k},\ldots,s'_{m},\theta'\rangle}  \tag{\emph{On-Ground}} \end{equation} where \emath{\alpha = Fail \wedge \beta= standby}
\end{boxD}
\label{op:sec}
 \caption{Operational Semantics Rules of the CSG Game Model.}
\end{figure*}


\subsection{Measure the efficacy of collaborative maneuvers}
Measuring the efficacy of collaborative maneuvers consists of synthesizing a strategy for players \emath{\mathcal{P}_{env}}, \emath{\mathcal{P}_{o}}, and \emath{\mathcal{P}_{g}} that has the objective of reaching a state-satisfying goal and maximizes the value of the reward. The specification for the synthesis of such strategy is given as rPATL property following the pattern \emath{\mathtt{  \langle\langle\textcolor{red}{\mathcal{P}_{env}}, \textcolor{red}{\mathcal{P}_{o}}, \textcolor{red}{\mathcal{P}_{g}}\rangle\rangle} \mathtt{ P=? [ F} \ goal\mathtt{]}} where \emath{goal=(}\quot{\emath{\textcolor{red}{win}}}\emath{\& \ \textcolor{red}{rounds}<=\textcolor{red}{k})} to quantitatively evaluate the efficacy of collaborative maneuvers to the round \emath{k}. However, to calculate the reward or cost related to collaborative maneuvers it will take the following pattern: \emath{\mathtt{  \langle\langle\textcolor{red}{\mathcal{P}_{env}}, \textcolor{red}{\mathcal{P}_{o}}, \textcolor{red}{\mathcal{P}_{g}}\rangle\rangle}} \emath{\mathtt{R\{}}\quot{\textcolor{red}{win}}\emath{{\}=?[F } \ goal\mathtt{]}} where \emath{goal} =(\emath{\textcolor{red}{rounds}<=\textcolor{red}{k}}). In this case, the reward reflects the number of times the collaborative maneuvers wins the game within a specific round  \emath{k.}
\end{sloppypar}

\section{Quantitative analysis using PRISM-games}\label{useCase}
\label{sec:useCase}
\begin{sloppypar}


\subsection{Players and system model}
The formal model of the satellite system will be divided into three Markov Decision Process (MDP) player models scheduled using a non-player module. The environment player, denoted by \emath{\mathcal{P}_{env}}, encapsulates the failure states, including both scheduled maintenance and unscheduled interruptions. The model is presented in \lst{CSGSatteliteModel}. The environment player encapsulates five commands: Lines 4 and 5 trigger an interruption in the normal state with probability \emath{1-e^{mtbf/t_{\alpha}}} (where MTBF is the mean time between failures and \emath{t_{\alpha}} is a temporary unavailability). Line 6 triggers a failure according to the satellite's reliability (\emath{1 - r}, where \emath{r} represents reliability). Both scheduled and unscheduled interruptions transition the satellite system back to its normal state. However, in case of failure, the reset is synchronized with the completion (success or termination) of maneuvers by ground or on-orbit staff.

\lstdefinestyle{framed}
{
	frame=lrb,         
	mathescape,
	numbers=left,
	belowcaptionskip=-1pt,
    xleftmargin=3.11em,
		xrightmargin=0.03cm,
    framexleftmargin=3em,
	framexrightmargin=0pt,
	framextopmargin=5pt,
	framexbottommargin=5pt,
	framesep=0pt,
	rulesep=0pt,
	numbers=left,
}
    
%\lstset{breaklines=true,style=framed,escapeinside={<@}{@>},
%	morekeywords={void, int, public,private,class,protected, submodules, network,connections, const, init,int,,bool, double, module,rewards,endrewards, endmodule},basicstyle=\scriptsize\ttfamily, keywordstyle=\bfseries\color{eminence}, 
%	morecomment=[f][\color{forestgreen}][0]{/*},
%    label=queueemodel
%}
\lstset{
    breaklines=true,
    style=framed,
    escapeinside={<@}{@>},
    morekeywords={void, int, public, private, class, protected, submodules, network, connections, const, init, int, bool, double, module, rewards, endrewards, endmodule},
    basicstyle=\scriptsize\ttfamily,
    keywordstyle=\bfseries\color{blue},
    morecomment=[f][\color{green!70!black}][0]{/*},
        morecomment=[l][\color{green!30!black}]{//},
    label=queueemodel
}


\begin{figure}[!htb]            
\begin{minipage}{9cm}
\begin{lstlisting}[style=framed,%customc,
	caption=Satellite Environmental Failure Model ,
 	label=CSGSatteliteModel]	
module connector
s3: [0..3] init 0;//0 NORMAL, 1 FAILURE, 2 SCHEDULED, 3 UNSCHEDULED
//interruptions and failures commands
[Scheduled]   s3=NORMAL ->  pow(exp,-mod(z,mtbf)/talpha):(s3'=NORMAL)	+(1-pow(exp,-mod(z,mtbf)/talpha)):(s3'=SCHEDULED) ;
[Unscheduled] s3=NORMAL ->  pow(exp,-mod(z,mtbf)/talpha):(s3'=NORMAL)+(1-pow(exp,-mod(z,mtbf)/talpha)):(s3'=UNSCHEDULED) ;
[Failure]     s3=NORMAL ->  r:(s3'=NORMAL )	+(1-r):(s3'=FAILURE) ;
[Normal]      s3=NORMAL | s3=SCHEDULED | s3=UNSCHEDULED-> (s3'=NORMAL);
[Reset]      s3=Failure-> (s3'=NORMAL);
endmodule
\end{lstlisting}
 \end{minipage}  
\end{figure}

The On-orbit player model, denoted by \emath{\mathcal{P}_{o}}, is detailed in \lst{CSGSatteliteModelOnOrbitManoeuvers}. The model encapsulates four commands that interpret global behavior from \fig{fig:usecase}. The first command (line 5) models a software update that transitions the satellite from standby status to update by modifying the variable \emath{s2.}A successful update is modeled by the command in line 8 with probability \emath{e^{mtbf/mttr}} (where MTTR represents the mean time to repair). In the worst case, ground control checks the satellite's availability and replaces it with probability \emath{P_{\beta}} (line 14). Upon successful update or failure, the satellite resets to standby (line 13).
  

\begin{figure}[!htb]            
\begin{minipage}{9cm}
\begin{lstlisting}[style=framed,%customc,
	caption=OnOrbit Satellite Manoeuvers,
 	label=CSGSatteliteModelOnOrbitManoeuvers]	
module OnOrbitManoeuvers
// State variable declaration with initial value
s2: [0..10] init 0;  // s2 is an integer variable ranging from 0 to 10, initialized to 0
// State transition - repair on-orbit software update
[RepairOnOrbitSoftwareUpdate]   	// Label for the transition
s2=STANDBY -> (s2'=UPDATE);       // The system transitions from standby (s2=STANDBY) to update state (s2'=UPDATE) for software update
// State transition - check for redundant satellite (from update state)
[CheckOnOrbitRedundantSatellite]        s2=UPDATE  -> (1-pow(exp,-mod(z,mtbf)/tmttr)):(s2'=CHECKSATTELITE)+(pow(exp,-mod(z,mtbf)/tmttr)):(s2'=STANDBY) ;
// State transition - move/replace redundant satellite (from check state)
[MoveReplaceRedundantSatellite]         s2=CHECKSATTELITE           -> Pbeta:(s2'=REPLACE)+(1-Pbeta):(s2'=ONGROUNDSPARE);
// State transition - reset to standby
[ResetOnOrbitManoeuvers]       		// Label for the transition
s2=ONGROUNDSPARE | s2=REPLACE | s2=UPDATE    -> (s2'=STANDBY);  // The system can reset to standby from various states (ONGROUNDSPARE, REPLACE, UPDATE) and sets s2' (next state) to standby
endmodule
\end{lstlisting}
 \end{minipage}  
\end{figure}

The On-ground player model, denoted by \emath{\mathcal{P}_{g}}, is detailed in  \lst{CSGSatteliteModelOnGroundManoeuvers}. The model encapsulates five commands that interpret global on-ground behavior from \fig{fig:usecase}. The first command (line 5) models a staff member transitioning the satellite from standby status to on-ground spare by modifying the variable \emath{s1}. Building a new satellite is modeled by the command in line 6 with probability \emath{1-e^{mtbf/t_{\gamma}}}(where \emath{t_{\gamma}}represents the mean-time to build a satellite). In this case, ground control initiates preparations for launching the new satellite. When the satellite is built (line 8, taking \emath{t_{\delta}} time), they prepare it for launch (line 10). Finally, when the satellite is ready for launch (line 12), the ground staff returns to standby mode with probability \emath{P_{n}}.
 

\begin{figure}[!htb]            
\begin{minipage}{9cm}
\begin{lstlisting}[style=framed,%customc,
	caption=On-Ground Satellite Manoeuvers,
 	label=CSGSatteliteModelOnGroundManoeuvers]	
module OnGroundManoeuvers
s1: [0..10] init 0; //refers to the on-ground modes
//The satellite on the ground is on standby mode
[OnGroundStandby] s1=STANDBY  -> (s1'=ONGROUNDSPARE);
//Building mode of a new satellite or initiate the launch
[CheckOnGroundSatelliteToBuild] s1=ONGROUNDSPARE  -> (1-pow(exp,-mod(z,mtbf)/tgamma)):(s1'=BUILD)+(pow(exp,-mod(z,mtbf)/tgamma)):(s1'=LAUNCH);
//build a new satellite or fail and  standby mode
[CheckOnGroundSatelliteToManufecture]   s1=BUILD  -> (1-pow(exp,-mod(z,mtbf)/tdelta)):(s1'=MANUFECTURE)+(pow(exp,-mod(z,mtbf)/tdelta)):(s1'=STANDBY);
// When the satellite is manufactured launch it
[BuildOnGroundSatellite] s1=MANUFECTURE    -> (s1'=LAUNCH);
//reset the on-ground operations
[ResetOnGroundManoeuvers] s1=LAUNCH -> (1-Pn):(s1'=LAUNCH)+(Pn):(s1'=STANDBY);
endmodule
\end{lstlisting}
 \end{minipage}  
\end{figure}

\paragraph{Experimental setup} We have encoded properties in rPATL formalism. PRISM-games model checker 3.0 \cite{Kwiatkowska2020} performs probabilistic verification. These experiments were conducted on a Ubuntu-I7 system equipped with 32GB RAM. Multiple engines can be selected (refer to documentation \cite{engines}) offering performance benefits for specific model structures. 

\paragraph{Artifacts} The source code for the experiments described in this section is publicly available on a GitHub repository\cite{seaa2024}. %The website provides comprehensive instructions on how to replicate the experiments.


\subsection{Properties of the modeled system as game goals}


We have identified the need to analyze satellite systems' reliability, availability, and maintainability (RAM) properties. Reliability refers to the satellite's ability to function without failures, considering scheduled and unscheduled interruptions. Maintainability reflects the ease with which the satellite can be repaired, with in-orbit repair being a key aspect.  The PRISM-games tool provides support for automated analysis of properties expressed in rPATL. So we express the system properties in natural language and then map them to the rPATL structure:

\subsubsection{\underline{Reliability}} 

\emph{What is the probability that a satellite will need to be replaced by a new one in 15 years, given a reliability of 0.80 regarding failures within 30 attempts?}

	   
	    \begin{resp}{}
             \begin{equation}
             \splitdfrac{\mathtt{  \langle\langle\textcolor{red}{\mathcal{P}_{env}}, \textcolor{red}{\mathcal{P}_{o}}, \textcolor{red}{\mathcal{P}_{g}}} \rangle\rangle}{ \\ \mathtt{ P=? [ F \ (\quot{\textcolor{red}{replace}} \ \& \ \textcolor{red}{rounds}<=\textcolor{red}{k}) ] ,k=\textcolor{blue}{1:30:1}}} 
        \label{eq1}
        \tag{PRO1}
    \end{equation}
        \end{resp}
        \normalsize

\subsubsection{\underline{Maintainability}} \emph{How many times would a satellite need to be replaced by a new one in 15 years, assuming a reliability of 0.80 regarding failures within 30 attempts?}

	   
	    \begin{resp}{ }
             \begin{equation}
             \splitdfrac{\mathtt{  \langle\langle\textcolor{red}{\mathcal{P}_{env}}, \textcolor{red}{\mathcal{P}_{o}}, \textcolor{red}{\mathcal{P}_{g}}} \rangle\rangle}{ \\ \mathtt{ R\{\quot{\textcolor{red}{replace}} \}=? [ F \  (\textcolor{red}{rounds}<=\textcolor{red}{k}) ] ,k=\textcolor{blue}{1:30:1}}} 
        \label{eq2}
        \tag{PRO2}
    \end{equation}
        \end{resp}
        \normalsize

\subsubsection{\underline{Availability}} \emph{What is the availability of the satellite in 15 years, assuming a reliability of 0.80 regarding failures within 30 attempts?}. We use T as a reference to the maximum number of rounds (attempts) to reach a successful replacement.

    \begin{resp}{ }
             \begin{equation}
             \splitdfrac{\mathtt{  \langle\langle\textcolor{red}{\mathcal{P}_{env}}, \textcolor{red}{\mathcal{P}_{o}}, \textcolor{red}{\mathcal{P}_{g}}} \rangle\rangle}{ \\ \mathtt{ R\{\quot{\textcolor{red}{replace}} \}=? [ F \  (\textcolor{red}{rounds}<=\textcolor{red}{k}) ]/T ,k=\textcolor{blue}{1:30:1}}} 
        \label{eq3}
        \tag{PRO3}
    \end{equation}
        \end{resp}
        \normalsize

\subsubsection{\underline{Maintenance cost}} \emph{What is the cost of the satellite replacement in 15 years, assuming a reliability of 0.80 regarding failures within 30 attempts?} 

    \begin{resp}{ }
             \begin{equation}
             \splitdfrac{\mathtt{  \langle\langle\textcolor{red}{\mathcal{P}_{env}}, \textcolor{red}{\mathcal{P}_{o}}, \textcolor{red}{\mathcal{P}_{g}}} \rangle\rangle}{ \\ \mathtt{ R\{\quot{\textcolor{red}{cost}} \}=? [ F \  (\textcolor{red}{rounds}<=\textcolor{red}{k}) ] ,k=\textcolor{blue}{1:30:1}}} 
        \label{eq4}
        \tag{PRO4}
    \end{equation}
        \end{resp}
        \normalsize

To address the properties mentioned above, we propose extending the model with a module to track the number of rounds synchronized with module actions. Additionally, we will incorporate a reward structure to model replacement times and costs.
        
First, we enhance the model by incorporating an integer constant as in \cite{KNPS19} and a module (see \lst{lst:rounds}) to keep track of the number of rounds. In this case, as the commands are unaffected by the players' choices, they are considered unlabeled with empty action. Consequently, these commands are executed regardless of the actions taken by the players. Furthermore, the module is deterministic due to the disjoint guards present in both commands. 

\begin{figure}[htbp]            
\begin{minipage}{9cm}
\begin{lstlisting}[style=framed,%customc, 
caption=Rounds Module,
 	label=lst:rounds]	
const k; // number of rounds
module rounds // module to count the rounds
 rounds : [0..k+1];
 [] rounds<=k -> (rounds'=rounds+1);
 [] rounds=k+1 -> true;
endmodule
\end{lstlisting}
 \end{minipage}  
\end{figure}

Second, we define five reward structures in \lst{lst:cost}. These structures correspond to the replacement times for in-orbit and on-ground operations (lines 3-6). They are divided into separate reward structures for in-orbit and on-ground replacements (lines 7-12). The cost of these replacements is modeled in lines 13-18. Here, c1 represents the cost of replacing a satellite in orbit, while c2 represents the cost of manufacturing a new satellite on the ground. \emph{Properties in \ref{eq2} and \ref{eq4} with tag name \quot{replace} and \quot{cost} can be replaced with on-orbit and on-ground replace and cost as rewards names in \lst{lst:cost}}.

\begin{figure}[htbp]            
\begin{minipage}{9cm}
\begin{lstlisting}[style=framed,%customc, 
caption=Reward Structures in PRISM,
 	label=lst:cost]	
const double c1; //Cost of replacing a satellite in orbit
const double c2; //Cost of manufacturing a new satellite on the ground
rewards "replace" // Reward for replacing a satellite on ground and on-orbit
  s2=REPLACE : 1;  // Replace satellite with a new one
  s1=MANUFECTURE : 1; // Manufacture a new satellite to replace satellite 
endrewards
rewards "replaceOnOrbit" // Reward for replacing a satellite on orbit
  s2 = REPLACE : 1; // Replace satellite with a new one
endrewards
rewards "replaceOnGround" // Reward for replacing a satellite on ground
  s1 = MANUFACTURE : 1; // Manufacture a new satellite to replace satellite 
endrewards
rewards "costOnOrbit" // Cost of replacing a satellite on orbit
  s2 = REPLACE : c1 * rounds; // Replace satellite with a new one, incurring a cost proportional to rounds
endrewards
rewards "costOnGround" // Cost of replacing a satellite on ground
  s1 = MANUFACTURE : c2 * rounds; // Manufacture a new satellite to replace satellite, incurring a cost proportional to rounds
endrewards
\end{lstlisting}
 \end{minipage}  
\end{figure}

\subsection{Experiments results analyses}

\noindent
   \begin{figure*}[htbp]
       \begin{tabularx}{\linewidth}{m{8cm}  m{8cm}}
 

 \begin{minipage}[t]{8cm}
     \centering

    \includegraphics[width=170pt, height =135pt]{Graph01.pdf}
    \caption{Verif. \ref{eq1} .}
    \label{fig:1}
   \end{minipage}
    
           &
           

 \begin{minipage}[t]{8cm}
     \centering

    \includegraphics[width=170pt, height =135pt]{Graph02.pdf}
    \caption{Verif. \ref{eq2}.}
    \label{fig:2}
   \end{minipage}
    

       \\ 
   \begin{minipage}[t]{8cm}
     \centering

    \includegraphics[width=170pt, height =135pt]{Graph03.pdf}
    \caption{Verif. \ref{eq3} .}
    \label{fig:3}
   \end{minipage}
    
           &
           

 \begin{minipage}[t]{8cm}
     \centering

    \includegraphics[width=170pt, height =135pt]{Graph04.pdf}
    \caption{Verif. \ref{eq4}.}
    \label{fig:4}
   \end{minipage}
   
           \end{tabularx}
\end{figure*}

The verification results for \ref{eq1} are shown in \fig{fig:1}. We observe that as the number of rounds increases, the probability of replacement for satellites in orbit or on the ground also increases. This is likely because the staff are more willing to replace satellites in multiple rounds to ensure maximum functionality and avoid failures throughout their 15-year lifespan.

The verification results for \ref{eq2} are shown in \fig{fig:2}. As the number of rounds increases, the probability of replacing both on-orbit and ground satellites also increases. However, on-orbit maintenance is more frequent than ground maintenance. This is due to the higher cost of manufacturing and launching a new satellite. Consequently, on-orbit staff tend to prioritize repairing existing satellites or finding spare ones in orbit whenever possible.

The verification results for \ref{eq3} are shown in \fig{fig:3}. We observe that the availability of the satellite increases. This can be interpreted as an increase in the satellite replacement rate as the number of rounds increases. This strategy aims to maximize the satellite's availability after 30 rounds. However, it's important to note that the maximum achievable replacement rate is equivalent to 0.008\%, which represents the limit of the staff's effort.


The verification results for Equation \ref{eq4} are shown in \fig{fig:4}. These results indicate that on-orbit staff maintenance incurs a cost of \$0.035 million more than on-ground maintenance. This is because staff prioritize on-orbit replacements over ground replacements. Interestingly, our initial assumption was that on-ground maintenance would be more expensive than on-orbit maintenance. However, the actual cost depends on the total number of replacements performed (as shown in \fig{fig:4}). This suggests that on-ground staff can develop a maintenance strategy that achieves higher satellite availability in fewer rounds, even though this strategy might require a larger initial maintenance budget.

\subsection{Threats to validity}
Our model currently focuses on a single satellite system. (i) For scenarios involving multiple cooperating satellites, additional staff collaboration would be necessary for repairs and maintenance, potentially introducing a first-in-first-out (FIFO) queueing system.  Furthermore, the maintenance parameters may change. (ii) The model does not consider communication between the ground station and the satellite system, which factors like solar radiation can impact. (iii) The interaction between on-orbit and ground staff has not been explicitly modeled in our current approach. However, their collaboration can significantly impact the maintenance strategy like the team size and involvement cost.


\subsection{Discussion}


Strategic maintenance implementation can enhance the quality of a satellite system while optimizing failure repair. The model can be extended with additional commands to simulate communication. Our current focus is on existing commands implemented by our collaborators and researchers.

The model is parametric, allowing for customization using different parameters within the PRISM-games. The data employed aligns closely with the models described in  \cite{Hoque2015,Yu2015}. However, a key difference lies in the model formalism. Previous implementations relied on CTMC and MDPs, whereas this work leverages the CSG formalism to include the human response. To our knowledge, this is the first implementation using CSGs for RAM analysis in satellite systems with collaborative maintenance (based on the PRISM library \cite{prismgames}).

Failures are primarily modeled using an exponential distribution. Additional distributions can be incorporated through appropriate model updates, reflecting the failure distribution of specific components.

\end{sloppypar}


\section{Conclusion}\label{conclusion}
\begin{sloppypar}
This paper demonstrates how a probabilistic model checker, specifically the PRISM-games, can be effectively used to model collaborative maintenance between on-orbit and ground staff. We evaluated the collaborative operation's effectiveness by performing a RAM (Reliability, Availability, and Maintainability) analysis on the satellite system. Previous research has primarily focused on improving model performance, neglecting the role of human intervention and response to failures.

Our future work will concentrate on analyzing maintenance strategies for constellations of satellite systems while ensuring a certain level of communication reliability between staff and equipment. This will ultimately lead to improved maintenance planning and investment optimization.
\end{sloppypar}

%\newpage
\bibliographystyle{unsrt}
{\scriptsize
\bibliography{references}}

\end{document}
\endinput
%%
%% End of file `sample-sigconf.tex'.