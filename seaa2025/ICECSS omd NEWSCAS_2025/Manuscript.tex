\documentclass[conference]{IEEEtran}
\IEEEoverridecommandlockouts
%
\usepackage{changepage}
\usepackage{mathtools}
\usepackage{graphicx}
\usepackage{xcolor}
\usepackage{tcolorbox}
\usepackage{makeidx}
\usepackage{float}
\usepackage{graphicx}
\usepackage{makeidx}  % allows for indexgeneration
\usepackage[english]{babel} % un troisième package
\usepackage{amssymb}
\usepackage{syntax}
\usepackage{multirow}
\usepackage{array}
%\usepackage[table]{xcolor}
\usepackage{colortbl}
\usepackage{enumitem}
\usepackage{booktabs}
\usepackage{listings}
%\usepackage{verbatim}
\usepackage{caption}
\usepackage{subcaption}
\usepackage{courier}
\usepackage{amsmath}
\usepackage{mathtools}
\usepackage{wasysym}
%\usepackage[sort=appearance]{spbasic}
%\usepackage{mathrsfs}
%\usepackage{sectsty}
%\usepackage{mathrsfs}
%\usepackage[numbers]{natbib}
%\usepackage[numbers,sort]{natbib}
%\usepackage[numbers]{natbib}
%\usepackage[backend=biber,style=numeric,sorting=none]{biblat‌ex}
%\bibliographystyle{unsrt}
%\setcitestyle{authoryear,open={(},close={)}}
%\setcitestyle{square,aysep={},yysep={;}}
%\usepackage[sort&compress]{natbib}
%\usepackage{amsthm}
%\usepackage{mathrsfs}
%\usepackage[authoryear]{natbib}
%\usepackage{amsthm}
\usepackage{tikz}
\usepackage[linesnumbered]{algorithm2e}
%\usepackage{algorithm}
%\usepackage{algorithmic}
\usepackage{algpseudocode}
\definecolor{bluekeywords}{rgb}{0.13, 0.13, 1}
\definecolor{greencomments}{rgb}{0, 0.5, 0}
\definecolor{redstrings}{rgb}{0.9, 0, 0}
\definecolor{graynumbers}{rgb}{0.5, 0.5, 0.5}
\definecolor{lightgray}{rgb}{0.83, 0.83, 0.83}
\definecolor{magnolia}{rgb}{0.93, 1.96, 1.3}


%\newtcolorbox{units}{before=\par\smallskip\centering,after=\par,hbox}

\definecolor{commentgreen}{RGB}{2,112,10}
\definecolor{eminence}{RGB}{108,48,130}
\definecolor{weborange}{RGB}{255,165,0}
\definecolor{frenchplum}{RGB}{129,20,83}
\definecolor{teagreen}{rgb}{0.82, 0.94, 0.75}
\definecolor{asparagus}{rgb}{0.53, 0.66, 0.42}
\usepackage[hyphens]{url}
\usepackage{hyperref}
\hypersetup{
  colorlinks,
  citecolor= 	blue,
  linkcolor= 	blue,
  urlcolor= 	blue}
    \usepackage{breakurl}
    \hypersetup{colorlinks=true,breaklinks=true}
  \usepackage{eucal}
  \usepackage{tabularx}
\usepackage{rotating}

\usepackage{listings}     
\usepackage{lstautogobble}  % Fix relative indenting
\usepackage{color}          % Code coloring
\usepackage{zi4}            % Nice font

\usepackage{lineno}



%\linenumbers


\DeclareCaptionFont{white}{\color{white}} 
\DeclareCaptionFormat{listing}{\colorbox{gray}{\parbox{\dimexpr\linewidth-1.9\fboxsep\relax}{#1#2#3}}}

\captionsetup[lstlisting]{format=listing,labelfont=white,textfont=white} 

\newcommand{\gtext}[1] {\textcolor{forestgreen}{#1}}


\newcommand{\keyw}[1] {\texttt{\textbf{#1}}}

\newcommand{\ie}{i.e., }
\definecolor{forestgreen}{rgb}{0.0, 0.5, 0.0}

\definecolor{aliceblue}{rgb}{0.94, 0.97, 1.0}



\newcommand{\eclipse}[1] {\textcolor{eminence}{\texttt{\textbf{#1}}}}

\newcommand{\num}[1] {\texttt{#1}}

\newcommand{\key}[1] {\texttt{\textbf{#1}}}


\newcommand{\sset}[1] {\llbracket #1 \rrbracket}

\newcommand{\eg}{e.g. }

\newcommand{\ai}{ Artificial Intelligence}

\newcommand{\ecircle}[1] {\textcircled{{\small #1}}}

\newcommand{\cmt}[1] {\textcolor{blue}{#1}}


\newcommand{\tab}[1]{Table \ref{#1}}

\newcommand{\algo}[1]{Algorithm \ref{#1}}

\newcommand{\commenting}[1] {\textcolor{blue}{#1}}

\newcommand{\quot}[1] {``#1''}

\newcommand{\emath}[1] {$#1$}

\newcommand{\emathtt}[1] {$\mathtt{#1}$}

\newcommand{\msym}[1] { $\mathscr{#1}$ }

\newcommand{\alg}[1] {Algorithm \ref{#1}}

\newcommand{\lst}[1] {Listing \ref{#1}}

\newcommand{\link}[2] {\texttt{#1}$\rightarrow$ \texttt{#2}}

\newtheorem{mydef}{\normalfont \textbf{Definition}}

\newcommand{\AB}[1] {\textcolor{red}{#1}}

\newcommand{\BH}[1] {\textcolor{blue}{#1}}

\newcommand{\gl}[1] {\langle {#1} \rangle}
\newcommand{\effect} {\textrm{Effect}}
\newcommand{\push} {\textrm{Push}}
\newcommand{\pull} {\textrm{Pull}}

\newcommand\myeq{\stackrel{\mathclap{\footnotesize\mbox{def}}}{=}}

% draw a frame around given text
\newcommand{\framedtext}[1]{%
\par%
\noindent\fbox{%
    \parbox{\dimexpr\linewidth-2\fboxsep-2\fboxrule}{#1}%
}%
}

\definecolor{commentgreen}{RGB}{2,112,10}
\definecolor{eminence}{RGB}{108,48,130}
\definecolor{weborange}{RGB}{255,165,0}
\definecolor{frenchplum}{RGB}{129,20,83}

\newtheorem{example}{\normalfont \textbf{Example}}  % Create the "Example" environment with plain style

\newcommand{\code}[1] {\texttt{#1}}

%\spnewtheorem{myexample}{Example}[section]{\bfseries}{\itshape}


\usepackage{tikz}
\usetikzlibrary{calc}
\usepackage{tcolorbox}

%%%%%%%%%%%%%%%%%%%%%%%%%%%%%%%%%%%%%%
\usetikzlibrary{calc,shadows.blur}
\tcbuselibrary{skins} % Assuming you have the "skins" library installed

\newtcolorbox{resp}[2][]{%
  enhanced jigsaw,
  colback=white,%
  colframe=gray,
  size=small, % Choose which "size=small" to keep
  boxrule=1pt,
  title=#2,
  halign title=flush center,
  coltitle=black,
  drop shadow=black!3!white,
  attach boxed title to top left={xshift=1cm,yshift=-\tcboxedtitleheight/2,yshifttext=-\tcboxedtitleheight/2},
  minipage boxed title=3cm,
  boxed title style={%
    colback=white,
    size=fbox, % Choose which "size=fbox" to keep
    boxrule=1pt,
    boxsep=2pt,
    underlay={%
      \coordinate (dotA) at ($(interior.west) + (-0.5pt,0)$);
      \coordinate (dotB) at ($(interior.east) + (0.5pt,0)$);
      \begin{scope}
        \clip (interior.north west) rectangle ([xshift=3ex]interior.east);
        \filldraw [white, blur shadow={shadow opacity=60, shadow yshift=-.75ex}, rounded corners=2pt] (interior.north west) rectangle (interior.south east);
      \end{scope}
      \begin{scope}[gray!80!black]
        \fill (dotA) circle (2pt);
        \fill (dotB) circle (2pt);
      \end{scope}
    },
  },
  #1, % Move "Learned" to before the closing curly brace
}%

\newtcolorbox{boxD}{
    colback = white, 
    colframe = black, 
    boxrule = 0pt, 
    toprule = 3pt, % top rule weight
    bottomrule = 3pt % bottom rule weight
}
\newtcolorbox{boxF}{
    colback = yellow!5!white,
    enhanced,
    boxrule = 1.5pt, 
    colframe = white, % making the base for dash line
    borderline = {1.1pt}{0pt}{main, dashed} % add "dashed" for dashed line
}

\newtcolorbox{boxC}{
    colback = blue!0!white,  % background color
    boxrule = 0pt  % no borders
}
\newcommand{\fig}[1]{Figure \ref{#1}}
%%%%%%%%%%%%%%%%%%%%%%%%%%%%%%%%%
% Exercise Environment
%\newenvironment{exercise}{\exerciseinner\mbox{}\par\bigskip}{\endexerciseinner}
%\newcommand{\theexercise}{\theexerciseinner}

% Exercise Environment
%\tcolorboxenvironment{exercise}{
%  breakable,
%   enhanced,
%   colback=gray!7!white,
%   parbox=false, drop fuzzy shadow
%}

%%%%%%%%%%%%%%%%%%%%%%%%%%
\DeclareCaptionFont{white}{\color{white}}
\DeclareCaptionFormat{listing}{%
    \colorbox{black}{\parbox{\dimexpr\textwidth-2\fboxsep}{\textbf{\textcolor{white}{#1#2#3}}}}}
\captionsetup[lstlisting]{format=listing,labelfont=white,textfont=white}


\def\llbracket{[\![}
\def\rrbracket{]\!]}

\newcommand{\CO}[1] {\langle\langle #1 \rangle\rangle}



\newcommand\setItemnumber[1]{\setcounter{enumi}{\numexpr#1-1\relax}}

\newcommand{\gparrow}[1] {\lhook\joinrel\xrightarrow{#1} }

\newcommand{\project} {  \url{https://hermes-design.github.io/ieaaie24.html} }


%\newcommand{\hermes} {\textbf{H}uman-\textbf{C}entric \textbf{C}ollaborative \textbf{A}rchitectural \textbf{D}ecision-\textbf{M}aking for \textbf{S}ecure \textbf{S}ystem \textbf{D}esign}

\newcommand{\hermes} {Human-Centric Collaborative Architectural Decision-Making for Secure System Design}


\usepackage{lineno}

\usepackage{calrsfs}
\DeclareMathAlphabet{\pazocal}{OMS}{zplm}{m}{n}
\newcommand{\La}{\mathcal{L}}
\newcommand{\Lb}{\pazocal{L}}

\def\BibTeX{{\rm B\kern-.05em{\sc i\kern-.025em b}\kern-.08em
    T\kern-.1667em\lower.7ex\hbox{E}\kern-.125emX}}

    
\begin{document}
\counterwithin{lstlisting}{section}
%\title{Applying Formal Verification to Assess Reliability, Availability, and Maintainability in Galileo Search and Human Rescue Services}
\title{Applying Formal Verification to Assess Galileo Search and Human Rescue Services}
\author{\IEEEauthorblockN{Abdelhakim Baouya}
\IEEEauthorblockA{\textit{IRIT, Université de Toulouse, CNRS, UT2} \\
118 Route de Narbonne, 31062 Toulouse Cedex 9, France \\
abdelhakim.baouya@irit.fr}
\and
\IEEEauthorblockN{Brahim Hamid}
\IEEEauthorblockA{\textit{IRIT, Université de Toulouse, CNRS, UT2} \\
118 Route de Narbonne, 31062 Toulouse Cedex 9, France  \\
brahim.hamid@irit.fr}
\and
\IEEEauthorblockN{Otmane Ait Mohamed}
\IEEEauthorblockA{\textit{Department of Electrical and Computer Engineering (ECE)} \\
Concordia University, Montréal, Canada \\
otmane.aitmohamed@concordia.ca}
\and
\IEEEauthorblockN{Saddek Bensalem}
\IEEEauthorblockA{\textit{University Grenoble Alpes, VERIMAG, Grenoble, France} \\
Grenoble, France\\
saddek.bensalem@univ-grenoble-alpes.fr}
}

\maketitle

\begin{abstract}
The Galileo Search and Rescue (SAR) system is a cornerstone of European maritime rescue operations. It leverages a network of satellites to pinpoint the location of individuals in distress. With a designed lifespan exceeding 12 years, the system relies on robust components with inherent dependability parameters such as Reliability, Availability, and Maintainability (RAM). These parameters are crucial for assessing the overall rescue performance in situations involving individuals in danger. This paper presents a novel approach utilizing Continuous-Time Markov Chains (CTMCs) and their formal specification in Continuous Stochastic Logic (CSL). We leverage the PRISM model checker for quantitative analysis, considering degradation scenarios within the communication elements and the evolving status of the distressed person. 
\end{abstract}

\begin{IEEEkeywords}
Galileo Search and Rescue, Reliability, Availability, Maintainability, PRISM
\end{IEEEkeywords}

\section{Introduction}
\label{introduction}
\begin{sloppypar}

Satellite systems have become a crucial part of our daily lives. The demand for reliable communication anywhere on Earth has driven innovation in the space industry, fostering competition among new space entrepreneurs like SpaceX and Amazon. These satellite constellations serve a wide range of purposes, including remote sensing for applications like the Internet of Things (IoT) and weather forecasting, as well as supporting critical military operations.  Therefore, Reliability, Availability, and Maintainability (RAM) are paramount considerations during the design phase.  A focus on RAM ensures systems are dependable and maintainable, minimizing the costs and complexities associated with potential repairs.

Formal verification \cite{Kwiatkowskaprism2011} is a powerful technique for ensuring the correctness and reliability of complex systems. It utilizes various formalisms, each suited to specific use cases. Some common formalisms include Markov Decision Processes (MDPs), Continuous-Time Markov Chains (CTMCs), and Concurrent Stochastic Games (CSGs). Stochastic games verification \cite{Kwiatkowska2019} allows for the generation of quantitative correctness assertions about a system's behavior (e.g. \quot{The object recognition system can correctly identify pedestrians with a probability of at least 95\%, even in challenging lighting conditions}.), where the required behavioral properties are expressed in quantitative extensions of temporal logic. The problem of strategy synthesis constructs an optimal strategy for a player, or coalition of players, to ensure a desired outcome (property) is achieved. The formalism of Concurrent stochastic multi-player games (CSGs) \cite{Kwiatkowska2019,Kwiatkowska2020} permits players to choose their actions concurrently in each state of the model. This approach captures the true essence of concurrent interaction, where agents make independent choices simultaneously without perfect knowledge of others' actions. However, although algorithms for verification and strategy synthesis of CSGs have been implemented in PRISM-games\cite{Kwiatkowska2021}, their adoption for RAM analysis has not been investigated.


This paper demonstrates how to accurately model satellite systems and verify their Reliability, Availability, and Maintainability (RAM) properties using the PRISM-games model checker. The PRISM-games tool extends the capabilities of classical probabilistic model checkers, which have been widely applied to verify the correctness and effectiveness of hardware and software designs \cite{prismmodelchecker}. However, classical models in probabilistic automata (PA) often focus on a single perspective.  Our approach leverages Concurrent Stochastic Games (CSGs) to capture the collaborative maintenance and repair of satellite systems \cite{Hoque2015,Yu2015,Zhaoguang2013}. This allows us to model multiple players involved in the system's operation, including the environment (which introduces failures and planned/unplanned interruptions), on-orbit maintenance maneuvers (responsible for software update moving to a redundant satellite), and ground control (responsible for satellite preparation and launch). CSGs are particularly well-suited for this task as they comprehensively represent failure management and collaborative maintenance.


\paragraph*{Outline}The remainder of this paper is structured as follows. Section~\ref{sec:rw} reviews related work. Section~\ref{Preliminaries} provides the necessary background on Concurrent Stochastic Games (CSGs) and the PRISM-games language. Section~\ref{sattelitemodel} presents our proposed modeling approach for satellite systems. We then perform a quantitative analysis of the satellite system's behavior under failure scenarios in Section~\ref{sec:useCase}. Finally, Section~\ref{conclusion} concludes the paper and suggests directions for future research.
\end{sloppypar}

\section{Related works}
\label{works}
\begin{sloppypar}
Several works in the literature have addressed the \cmt{quality and deployment} of satellite systems. In \cite{Zhaoguang2013}, the authors propose \cmt{to use CTMC to model} satellite systems, demonstrating how this approach can analyze the impact of \cmt{different} factors, \cmt{such as} solar radiation, on satellite reliability and maintenance. Building upon this work, \cite{Hoque2015} incorporates Erlang distributions into the CTMC framework to further refine the modeling of maintenance and scrubbing activities. Extending these studies, \cite{Zhaoguang2016} focuses on modeling the reliability of an entire satellite constellation using CTMCs. \cmt{Additionally}, \cite{Baouyaseaa2024} \cmt{broadness} the scope by incorporating human \cmt{capabilities} in maintaining satellite systems within a \cmt{game-theoritical} model.

While \cmt{significant} research has focused on the quality and deployment of US GPS satellites, research activities \cmt{explicitely} addressing the reliability and performance of the SAR/Galileo satellite system are less prevalent in the literature. \cmt{Known studies addressing this topic} include \cite{Alegre2014,Inone2027,Lewandowski2008}. In \cite{Lewandowski2008}, the authors investigate a hybrid communication infrastructure combining terrestrial networks with enhanced Galileo SAR for firefighter missions. Using simulations, they compare the advanced Galileo SAR system with the existing Cospas-Sarsat system. They identify performance metrics and determine the optimal number of Galileo satellites to enhance SAR service quality and minimize emergency response times.  The authors in \cite{9115548} evaluate the performance of four Galileo satellites launched in 2018 during their first year of operational service ( to reach 22 satellites). Key performance indicators, including signal-in-space status, availability, and ranging accuracy, were analyzed. Results demonstrate high signal health and availability, with minimal signal interruptions. Moreover, the numerical results demonstrate high reliability for the most recent additions to the Galileo satellite constellation. The authors in \cite{Inone2027} focus on the development of the SAR/Galileo ground system in Korea, outlining a roadmap strategy for the next-generation search and rescue system with the goal of enhancing rescue efficiency by improving location accuracy and reducing response times. The authors in \cite{Alegre2014} investigate methods to improve the availability of the Return Link Service (RLS) in the Galileo SAR system. \cmt{Their} proposed solutions, \cmt{utilize} Network Coding, \cmt{to improve} RLM reception by mitigating signal losses due to harsh weather or \cmt{obstacles and} considering backward compatibility and system complexity. \cmt{However, none} of the reviewed research explicitly addresses the quality of Galileo SAR services from the perspective of reliable intervention, \cmt{especially regarding} the dependability parameters \cmt{mentioned} in the documentation and the psychological status of the person in distress.
\end{sloppypar}

\section{Preliminaries}
\label{preliminaries}
\begin{sloppypar}
Probabilistic Model Checking using PRISM \cite{Kwiatkowskaprism2011} relies on constructing a formal model, typically represented using appropriate storage structures. The verification process is then performed by applying a suite of algorithms implemented within the PRISM engine \cite{engines}. For our analysis, we employ Continuous-Time Markov Chains (CTMCs) \cite{Kwiatkowska2007}, a well-established modeling technique for evaluating reliability and performance. The CTMC involves a set of states and a transition matrix \emath{\textbf{R}: S \times S \rightarrow \mathbb{R}_{\geq 0}}.  The rate specifies the delay before a transition between states s and s' takes place \emath{\textbf{R}(s,s')}, where the probability between \emath{s} and \emath{s'} take within time t is given by the value \emath{1-e^{-\textbf{R}(s,s') \times t}}. Based on research using PRISM for CTMC modeling, as outlined in \cite{prismctmc}, exponentially distributed delays are often considered suitable for modeling electronic component lifetimes and inter-arrival times. 

The PRISM model is composed of a set of modules that can synchronize. Each module is characterized by variables and commands (or transitions). The valuations of these variables represent the state of the module. The behavior of each module is described using a set of commands, each of which follows the following format:
\[
[a] \ g \ \rightarrow \ \lambda: u
\]
This indicates that if the guard condition \( g \) evaluates to true, then the update \( u \) is enabled to occur with a rate of \( \lambda \) for action \( a \). A guard is a boolean formula constructed from the module variables. The update \( u \) is an evaluation of variables expressed as a conjunction of assignments: \emath{v_{i}' = val_{i} + \ldots + v_{n}' = val_{n}} where \( v_{i} \in V \), with \( V \) being a set of local and global variables, and \( val_{i} \) are values evaluated via expressions denoted by \( \theta \) such that \( \theta: V \rightarrow \mathbb{D} \), where \( \mathbb{D} \) is the domain of the variables.

Two types of reward functions are highlighted. The action reward function assigns a real value to each state-action. This value is accumulated when the action \( a \) is selected in the state \( s \). Additionally, the state reward function, denoted as \( r_{S} : S \longrightarrow \mathbb{R} \), assigns a real value to each state \( s \). This value is accumulated when the state \( s \) is reached.

Properties are typically expressed in Continuous Stochastic Logic (CSL) \cite{kwiatkowska2002approximate}, a stochastic variant of the well-known Computational Tree Logic (CTL).  For instance, the following property expressed in natural language: \emph{Is the probability of that eventually the system failure occurring within 100 time units is less than 0.001} is expressed  as: \emath{ P_{<0.001} [ F^{\leq 100} \ fail ]}
Here, \(fail\) is the label that refers to the system failure states. 
Regarding the reward structure, the property expressed in natural language: \emph{What is the amount of reward accumulated over a specific 100 times units ?} is expressed in CSL as:
\emath{ R\{"up"\}=? [C \leq 100]}.


\end{sloppypar}


\section{Formal modeling of satellite systems}
\label{sattelitemodel}
\begin{sloppypar}

\begin{figure}[htbp]
     \centering
     		\includegraphics[width=320pt, height =200pt]{cs.pdf}
     \caption{SAR/Galileo Services \cite{galileoperformances,galileoossdd}.}
     \label{fig:usecase}
 \end{figure} 

\fig{fig:usecase} depicts the SAR/Galileo Services, illustrating the organizational structure of rescue services. The system encompasses multiple components that relay beacon signals from distressed users to rescue authorities. This section presents the system parameters and the formal response mode model.

\subsection{The system model}


COSPAS-SARSAT (C/S) is an international satellite-based Search and Rescue (SAR) distress alerting system established in 1979 by the USA, Canada, France, and the former USSR \cite{galileosarservice}.(See \fig{fig:usecase}). The full documentation is available in \cite{galileosarsdd}. The C/S system comprises:

\begin{enumerate}
    \item Beacons are 406 MHz radio transmitting devices employed in various applications. These include Emergency Position Indicating Radio Beacons (EPIRBs) and Ship Security Alert Systems (SSAS) for maritime use, Emergency Locator Transmitters (ELTs) and ELT Distress Tracking for aviation, and Personal Locator Beacons (PLBs) for individual use.

    \item The space segment \cmt{includes} satellites operating in low Earth orbit, geostationary orbit, and medium Earth orbit, responsible for processing signals transmitted by beacons.

    \item A Service Ground Segment (SGS) \cmt{consists of} a geographically distributed set of receiving ground stations known as Local User Terminals (LUTs). This network provides ground segment coverage, \cmt{allowing} the tracking of satellites and the generation of independent location estimates for user beacons.

    \item Mission Control Centers (MCC) are crucial in distributing C/S distress alerts globally and configuring the alerts for optimal response.

    \item The Sar/Galileo Ground Segment \cmt{consists of} five Reference Beacons (REFBE). These reference beacons \cmt{distibuted across} the European coverage Area are used to monitor the performance of the SAR/Galileo Service.
\end{enumerate}

The Galileo Search and Rescue (SAR) Forward link service is capable of receiving signals emitted by C/S compatible 406 MHz distress beacons (Forward Link Alert Message, as FLAM)and relaying this information to a ground segment network, known as the Local User Terminal (LUT), which consists of geographically distributed facilities deployed worldwide. The Return Link Service (RLS) enables the relay of data (Return Link Messages, or RLMs) back to the originating beacon. A primary function of the RLS is to provide the end-user of a distress beacon with automatic acknowledgment, confirming the detection of the alert and the determination of their location by the Search and Rescue (C/S) system.

Upon estimating the beacon's location, the Mission Control Center (MCC) issues an RLM\_Request. This request, covering the beacon's confirmed position, is transmitted to a backup MCC (Spanish or French). Subsequently, the RLS generates an RLM Transmission Request (RLMR) based on the original FLAM. The Galileo Core infrastructure processes the RLMR and uplinks the RLM to \cmt{appriopriate} Galileo \cmt{satellites, wich then} is broadcast to the originating beacon.

\cmt{Once the beacon receives the RLM}, sets \cmt{a} receipt status flag within FLAM. This status flag is then transmitted to the RLSP via the C/S MCC. \cmt{After} the RLSP acknowledges the receipt of the RLM, the Galileo system \cmt{stops sending} further \cmt{RLMs} to the beacon. However, if no acknowledgment of RLM reception is received within 24 hours of the initial RLM request, the Galileo system \cmt{will continue} to transmit RLMs to the beacon.

\subsection{Operational capabilities characteristics}



According to \cite{galileoperformances} and \cite{galileoossdd}, Galileo satellites \cmt{demonstrate} a reliability exceeding 88\% over a 12-year lifespan \cmt{, with} an availability of 99.5\%. \cmt{The work in} \cite{galileoosperformancereport} \cmt{reports that the availability of healthy signals from Galileo is 99.22}\%, with a recovery time of less than 15 hours. \cmt{This emphasizes the importance of ensuring timely service recovery}. Detection performance, a \cmt{crucial} aspect of the system, \cmt{refers to} the probability of successfully detecting 406 MHz beacon transmissions within the SAR/Galileo coverage area and receiving a valid beacon message at the SAR/Galileo LUT Facilities.

Three service states are defined for the SAR Forward link service in the ECA (European Coverage Area): Nominal, Degraded, and Severely Degraded. Nominal indicates normal operation. Degraded is characterized by either non-operational status for less than 24 hours continuously or less than 48 hours cumulatively over a calendar month or by losing communication with one or two ground segments (LUTs) for more than one day, or only 1 REFBE is in nominal status, or no REFBE is operational for less than 5 continuous days. Severely Degraded occurs when the SGS is non-operational for more than 24 hours continuously or more than 48 hours cumulatively over a calendar month or when communication is lost with all three LUTs and MCCs for more than four hours.

Similarly, the RLS has three service states: Nominal, Degraded, and Severely Degraded. The system is considered \quot{Degraded} if the RLSP is in degraded status or not operational for up to 7 cumulative hours within a calendar month or if RLM messages are delivered but not compliant with the latency MPL: The RLM Delivery Latency within 15 min was above or equal to 99.95\% \cite{galileoperformances}) which means that the Failure Rate is 0.05\%, and thus MTTF  is 2000 hours. \quot{Severely Degraded} status applies when the RLSP is not operational for more than 7 cumulative hours within a calendar month or if RLM messages are not delivered for more than 7 hours. In addition, based on the given availability of 99.8\% and the assumption of a very high MTBF, the estimated MTTR for the RLS would be approximately 500 hours.

\subsection{Formal modeling}

\begin{figure}[htbp]
     \centering
     		\includegraphics[width=250pt, height =80pt]{automat.pdf}
     \caption{Markov Model Pattern for SAR/Galileo Services.}
     \label{fig:model}
 \end{figure} 
 
The SAR/Galileo system can be modeled as a three-state Markov chain, illustrated in \fig{fig:model} and portrayed in the PRISM textual representation in \lst{exampleinprismtest}. State  \emath{S_{1}} represents the fully operational system, \cmt{while state} \emath{S_{2}} \cmt{indicates} a faulty state requiring recovery. State \emath{S_{3}} \cmt{signifies} a severe degradation state where maintenance is required. In the PRISM model, the system states are represented by an integer variable \emath{S}, which can take values from 0 to 3 (line 13). The assignment of degradation status relies on the constant values defined in lines 2-5. In this model, \emath{\lambda} represents the failure rate of system components, and \emath{\mu_{1}} \cmt{indicates} the recovery with repair rate. We \cmt{assume that the rates of} constant degradation and severe degradation rates (failure) follow a Poisson process. In \lst{exampleinprismtest}, the parameters are parametrizable as defined in lines 8-10. The reference documentation includes \cmt{various} sources of failure; \cmt{allowing the} model to be parametrized \cmt{so that users can} select which failures \cmt{impact} the communication system. \cmt{Additionally, this paper discusses how human reliability in sending rescue signals is} influenced by their psychological status, which we also model using a Poisson distribution. 



\lstdefinestyle{framed}
{
	frame=lrb,
	mathescape,
	numbers=left,
	belowcaptionskip=-1pt,
	xleftmargin=3.11em,
	xrightmargin=0.03cm,
	framexleftmargin=3em,
	framexrightmargin=0pt,
	framextopmargin=5pt,
	framexbottommargin=5pt,
	framesep=0pt,
	rulesep=0pt,
	numbers=left,
}

\lstset{
    breaklines=true,
    style=framed,
    escapeinside={<@}{@>},
    morekeywords={void, int, public, private, class, protected, submodules, network, connections, const, init, int, bool, double, module, rewards, endrewards, endmodule,label,ctmc},
    basicstyle=\small\ttfamily,
    keywordstyle=\bfseries\color{blue},
    morecomment=[f][\color{green!70!black}][0]{/*},
    morecomment=[l][\color{green!30!black}]{//},
    label=queueemodel
}

\begin{figure}[!htb]
\begin{minipage}{12.3cm}
\begin{lstlisting}[style=framed,
	caption=The System Status Over Execution Time,
 	label=exampleinprismtest]
ctmc    
//System states
const int Operational = 2;
const int Degraded = 1;
const int Severely_Degraded = 0;

//System performance rates
const double $\lambda_{1}$;
const double $\lambda_{2}$;
const double $\mu_{1}$;
//PRISM module
module Degradation
 S : [0..3] init Operational;
 [degradation]       S>0       $\rightarrow \lambda_{1}$:(S'=Degraded); 
 [sever_degradation] S>0       $\rightarrow \lambda_{2}$:(S'=Severely_Degraded); 
 [reset]             S=Degraded$  \rightarrow \mu_{1}$:(S'=Operational); 
endmodule

\end{lstlisting}
\end{minipage}
\end{figure}

\end{sloppypar}


\section{Quantitative analysis using PRISM}
\label{usecase}
\begin{sloppypar}
In this work, we \cmt{focus on} in evaluating signal transmission \cmt{performance using the PRISM tool}, \cmt{explicitly} focusing on scenarios where the individual experiencing distress maintains a \cmt{stable} psychological status. 

\paragraph{Experimental setup.} We have encoded properties in CSL formalism \cmt{and utilized the} PRISM model checker v4.8 \cite{Kwiatkowska2020} is utilized \cmt{for} verification. These experiments were conducted on a \cmt{system running Ubuntu with an i7 processor and equipped with 32GB RAM}. Multiple engines can be selected (refer to documentation \cite{engines}) offering performance benefits \cmt{to} specific model structures. In addition, we have implemented the scenarios outlined in \cite{Zhu2009} to accurately model attack frequencies.

\paragraph{Artifacts.} The source code for the experiments described in this section is publicly available on a GitHub repository \cite{newcas2025}. The website provides \cmt{detailed} instructions on \cmt{for replicating} the experiments.

	    \begin{resp}{\textbf{\textit{Property 1}}}
        
        \begin{equation}
        \label{eq1}\tag{\emph{\quot{Liveness}}}
         \mathtt{ P=? [ G(\quot{\textcolor{red}{up} } \xrightarrow{} \  F \ !(\quot{\textcolor{red}{up} }))  ]} 
        \end{equation}

        
        \end{resp}
        
        \normalsize

The property \ref{eq1} evaluates the signal performance by calculating the probability of the satellite successfully collecting the signal from a beacon while facing degradation of the signal, represented by the label \emath{!\quot{\textcolor{red}{up} }}. The results indicate a 100\% probability of the system transitioning from a functioning state to a degraded state. Thus, this leads us to investigate the role of degradation features on the signal transmitted through the verification of properties \ref{eq2} and \ref{eq3}.


    \begin{figure}[!htb]
    \centering
       \begin{tabularx}{\linewidth}{ m{8cm} }
           

 \begin{minipage}[t]{12cm}
     \centering

    \includegraphics[width=200pt, height =180pt]{gdegraded.pdf}
    \caption{Verification of Property \ref{eq2}.}
    \label{fig:01}
   \end{minipage}
    
          \\

   \begin{minipage}[t]{12cm}
     \centering
   		\includegraphics[width=200pt, height =180pt]{gseverlydegraded.pdf}
    \caption{Verification of Property \ref{eq3}.}
    \label{fig:02}
   \end{minipage}

               \end{tabularx}
\end{figure}


	    \begin{resp}{\textbf{\textit{Property 2}}}
        
        \begin{equation}
        \label{eq2}\tag{\emph{\quot{Degraded}}}
         \mathtt{ P=? [ (\quot{\textcolor{red}{up} }) \  U^{\leq T} \ (\quot{\textcolor{red}{Degraded} })  ]} 
        \end{equation}
        
        \end{resp}
        
        \normalsize

The property \ref{eq2} \cmt{specifies} that as time elapsed between the nominal functioning state and the degradation state increases. However, the primary distinction lies in the specific feature causing the degradation. Figure \ref{fig:01} illustrates that after 10 hours of signal transmission, the probability of degradation reaches 50\% and persists at that level for 30 calendar days due to REFBE unavailability. In contrast, internal degradation within 24 hours exhibits a low probability of 3.1\%. Notably, combining internal degradation and communication loss with ECA results in a degradation probability of 5.4\%. When only one REFBE is operational, the degradation probability reaches 11.6\%  after 20 hours of execution.


	    \begin{resp}{\textbf{\textit{Property 3}}}

        \begin{equation}
        \label{eq3}\tag{\emph{\quot{Severely Degraded}}}
         \mathtt{ P=? [ (\quot{\textcolor{red}{up} }) \  U^{\leq T} \ (\quot{\textcolor{red}{Severely\_Degraded} })  ]} 
        \end{equation}
        
        \end{resp}
        
        \normalsize


        
The property \ref{eq3} \cmt{describes} that as time elapsed between the nominal functioning state and the severe degradation state increases. However, the primary distinction lies in the specific feature causing the degradation,  \cmt{similar to the \ref{eq3} property}. Figure \ref{fig:02} illustrates that after 10 hours of signal transmission, the probability of degradation reaches 50\% and persists at that level for 30 calendar days due to REFBE unavailability. In contrast, internal degradation within 24 hours and 48 hours exhibits a low probability of 3.1\% and 5.4\%, respectively.  When communication loss with ECA, the severe degradation probability reaches 1.9\% less than the degraded mode after 20 hours of execution.

Through our analysis, we have \cmt{pinpointed} the source of service degradation within the SAR/Galileo system. \cmt{Specifically}, issues related to REFBE (Reference Beacon) monitoring of satellite performance can significantly contribute to service degradation. These \cmt{problems} can also provide maintainers with erroneous fault results, impacting maintenance activities and quality assurance.



\subsection{Discussion}
\cmt{In this study, we examined} the performance of the SAR/Galileo system, specifically focusing on degradation issues \cmt{affecting} communication between satellites and the ground stations \cmt{that assist individuals} in distress. The use case is \cmt{based on} existing documentation, and we \cmt{aim} to \cmt{evaluate} the system's \cmt{capability} to save lives by \cmt{assessing} the \cmt{communication availability of each entity involved in transmitting signals.} \cmt{Although} the signal \cmt{includes} multiple parameters not explicitly \cmt{covered} in this study, our \cmt{emphasis} remains on a high-level perspective.

The model can be extended to incorporate other factors influencing satellite signal quality, such as the impact of solar radiation, as discussed in \cite{Hoque2015,baouya2024seaa}, which can significantly affect system reliability. Furthermore, human factors should also be considered, such as an individual's \cmt{ability} to push the distress button in an \cmt{emergency promptly}. 

These results do not demonstrate the possibility that the system does not fully encompass the state of the human in terms of their reaction to danger, such as the Mean  Time To React to danger (MTTR). To address this limitation, we augment the model with a module that represents the human's status in response to a crisis in a one-month calendar. This module incorporates a formula for the rate of urgent response in line 2 of \lst{exampleinprism}: 

\begin{equation*}
\lambda_{h} = MTTR / Month 
\end{equation*}

Consequently, the model is augmented to incorporate human behavior, as illustrated in \lst{exampleinprism}. Notably, the label \quot{\textcolor{red}{\emathtt{up}}} encompasses potential human ability degradation. The PRISM command depicts a state transition from the operational mode to a degraded status for the human, upon the value of the degradation rate parameter \emath{\lambda_{h}} in line 6 of \lst{exampleinprism}. However, the correct status of the communication service depends on the system's status, which can include nominal operation, degradation, severe degradation, and the degradation of human factors in line 9.

\lstdefinestyle{framed}
{
	frame=lrb,
	mathescape,
	numbers=left,
	belowcaptionskip=-1pt,
	xleftmargin=3.11em,
	xrightmargin=0.03cm,
	framexleftmargin=3em,
	framexrightmargin=0pt,
	framextopmargin=5pt,
	framexbottommargin=5pt,
	framesep=0pt,
	rulesep=0pt,
	numbers=left,
}

\lstset{
    breaklines=true,
    style=framed,
    escapeinside={<@}{@>},
    morekeywords={void, int, public, private, class, protected, submodules, network, connections, const, init, int, bool, double, module, rewards, endrewards, endmodule,label},
    basicstyle=\small\ttfamily,
    keywordstyle=\bfseries\color{blue},
    morecomment=[f][\color{green!70!black}][0]{/*},
    morecomment=[l][\color{green!30!black}]{//},
    label=queueemodel
}

\begin{figure}[!htb]
\begin{minipage}{12cm}
\begin{lstlisting}[style=framed,
	caption=The Human Status,
 	label=exampleinprism]
const double time_to_react;
const double lambda_h=time_to_react/(30*24);

module Human_Status
 Human_Status_s : [0..3] init Operational;
 [] Human_Status_s>0 -> $\lambda_h$:(Human_Status_s'=Degraded); 
endmodule

label "up" = !degraded & !Severely_Degraded & !(Human_Status_s=Degraded);
\end{lstlisting}
\end{minipage}
\end{figure}

	    \begin{resp}{\textbf{\textit{Property 4}}}
               \begin{equation}
        \label{eq4}\tag{\quot{Rescue Service}}
         \mathtt{ P=? [ (\quot{\textcolor{red}{up} }) \  U^{\leq T} \ !(\quot{\textcolor{red}{up} })  ]} 
        \end{equation}
        \end{resp}
        
        \normalsize

        
By examining Property \ref{eq4}, which integrates human status, we observe that the system's ability to rescue a person in distress diminishes as the execution time increases, as depicted in \fig{fig:03}, regardless of the Mean Time To React (MTTR). This decrease in rescue success is attributed to the increasing likelihood of the individual experiencing a psychological state that hinders their ability to activate the distress button.

While this may seem intuitive, the verification process mathematically confirms this observation. Furthermore, the results demonstrate that the system's effectiveness depends on its capability to rescue the person in danger and crucially relies on the person's timely response to the crisis.



\begin{figure}[htbp]
     \centering
   		\includegraphics[width=250pt, height =180pt]{Graphh.pdf}
    \caption{Verification of Property \ref{eq4} with varied MTTR.}
    \label{fig:03}
 \end{figure} 

\subsection{Threats to validity}

This paper \cmt{focuses on specific} operational parameters of the SAR/Galileo system. \cmt{Although} other parameters \cmt{mentionned} in this SAR/Galileo documentation could \cmt{also be verified}, the PRISM model checker has limitations in supporting particular formalisms, \cmt{particulary when} incorporating satellite location and accuracy using complex data representation. These parameters necessitate high-level language specifications. \cmt{The} BIP \cite{basurigorous2011} (Behavior-Interaction-Priority) language can \cmt{effectively represent} these parameters and \cmt{facilitate} verification using a dedicated statistical model checking with SMC-BIP \cite{med2018}. 
\end{sloppypar}

\section{Conclusion}
\label{conclusion}
\begin{sloppypar}
This paper demonstrates how a probabilistic model checker, specifically the PRISM-games, can be effectively used to model collaborative maintenance between on-orbit and ground staff. We evaluated the collaborative operation's effectiveness by performing a RAM (Reliability, Availability, and Maintainability) analysis on the satellite system. Previous research has primarily focused on improving model performance, neglecting the role of human intervention and response to failures.

Our future work will concentrate on analyzing maintenance strategies for constellations of satellite systems while ensuring a certain level of communication reliability between staff and equipment. This will ultimately lead to improved maintenance planning and investment optimization.
\end{sloppypar}

\newpage
% \section*{Acknowledgements}
% The research leading to the presented results was conducted within the research profile of \hermes~(HERMES-Design\footnote{HERMES-Design: \url{https://hermes-design.github.io/}}) supported by Institut de Cybersécurité d'Occitanie (ICO).

\bibliographystyle{unsrt}
{%\scriptsize
\bibliography{references}}

% \bibliographystyle{IEEEtranS}
% {%\scriptsize
% \bibliography{references}}

\end{document}