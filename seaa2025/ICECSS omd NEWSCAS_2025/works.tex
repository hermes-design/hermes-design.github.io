Several works in the literature have addressed the \cmt{quality and deployment} of satellite systems. In \cite{Zhaoguang2013}, the authors propose \cmt{to use CTMC to model} satellite systems, demonstrating how this approach can analyze the impact of \cmt{different} factors, \cmt{such as} solar radiation, on satellite reliability and maintenance. Building upon this work, \cite{Hoque2015} incorporates Erlang distributions into the CTMC framework to further refine the modeling of maintenance and scrubbing activities. Extending these studies, \cite{Zhaoguang2016} focuses on modeling the reliability of an entire satellite constellation using CTMCs. \cmt{Additionally}, \cite{Baouyaseaa2024} \cmt{broadness} the scope by incorporating human \cmt{capabilities} in maintaining satellite systems within a \cmt{game-theoritical} model.

While \cmt{significant} research has focused on the quality and deployment of US GPS satellites, research activities \cmt{explicitely} addressing the reliability and performance of the SAR/Galileo satellite system are less prevalent in the literature. \cmt{Known studies addressing this topic} include \cite{Alegre2014,Inone2027,Lewandowski2008}. In \cite{Lewandowski2008}, the authors investigate a hybrid communication infrastructure combining terrestrial networks with enhanced Galileo SAR for firefighter missions. Using simulations, they compare the advanced Galileo SAR system with the existing Cospas-Sarsat system. They identify performance metrics and determine the optimal number of Galileo satellites to enhance SAR service quality and minimize emergency response times.  The authors in \cite{9115548} evaluate the performance of four Galileo satellites launched in 2018 during their first year of operational service ( to reach 22 satellites). Key performance indicators, including signal-in-space status, availability, and ranging accuracy, were analyzed. Results demonstrate high signal health and availability, with minimal signal interruptions. Moreover, the numerical results demonstrate high reliability for the most recent additions to the Galileo satellite constellation. The authors in \cite{Inone2027} focus on the development of the SAR/Galileo ground system in Korea, outlining a roadmap strategy for the next-generation search and rescue system with the goal of enhancing rescue efficiency by improving location accuracy and reducing response times. The authors in \cite{Alegre2014} investigate methods to improve the availability of the Return Link Service (RLS) in the Galileo SAR system. \cmt{Their} proposed solutions, \cmt{utilize} Network Coding, \cmt{to improve} RLM reception by mitigating signal losses due to harsh weather or \cmt{obstacles and} considering backward compatibility and system complexity. \cmt{However, none} of the reviewed research explicitly addresses the quality of Galileo SAR services from the perspective of reliable intervention, \cmt{especially regarding} the dependability parameters \cmt{mentioned} in the documentation and the psychological status of the person in distress.