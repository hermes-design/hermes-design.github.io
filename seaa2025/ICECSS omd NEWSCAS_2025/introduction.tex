
Satellite systems have become \cmt{necessary} in modern life, fulfilling \cmt{an essential} role in \cmt{various} sectors, including maritime \cite{FRAIRE2024110874} and military operations \cite{NORRIS201144}. The increasing global demand for reliable communication has driven significant innovation within the space industry \cite{economist2024}. Given the critical nature of these systems, dependability assurance is \cmt{paramount, including} the performance assessment of operational tasks, particularly those related to human \cmt{rescues,} such as search and rescue (SAR) missions \cite{galileoperformances,galileoossdd,galileoosperformancereport}.


%Formal verification \cite{Kwiatkowskaprism2011} is a powerful technique for performance and dependability assessment of critical and complex systems. It focuses on probabilistic model checking, which involves constructing and analyzing probabilistic models, typically Markov chains or Markov processes \cite{baierprinciples2008}.
\cmt{Formal verification \cite{baierprinciples2008,Kwiatkowskaprism2011}  is a powerful technique that allows for thorough system analysis to prove the properties that ensure its correct operation.} \cmt{Different} formalisms can be \cmt{used}, each \cmt{suited} for specific \cmt{applications}. \cmt{Typical} examples include Markov Decision Processes (MDPs), Continuous-Time Markov Chains (CTMCs), and Concurrent Stochastic Games (CSGs). In contrast to simulation, which relies on analyzing results from \cmt{many} random samples, formal verification provides a mathematically rigorous and exhaustive analysis of the system's behavior.


\cmt{Our work in this} paper \cmt{illustrates} how to \cmt{effectively} model communication efficiency and verify its performance using the PRISM model checker  \cite{Kwiatkowskaprism2011}. \cmt{Although} the PRISM probabilistic model checker has been widely applied to verify the correctness and effectiveness of hardware and software designs \cite{prismmodelchecker}, its application to the specific context of SAR systems has been limited. Previous research, \cmt{including studies} \cite{Hoque2015,Yu2015,Zhaoguang2013,Baouyaseaa2024}, has \cmt{primarily concentrated} on \cmt{assessing} the dependability of the satellite itself (e.g., the US GPS Satellite). \cmt{In contrast, this} work focuses on the performance of communication services \cmt{during} request-response interactions between a person in distress and the Galileo satellite system. The Galileo satellite \cmt{handles requests}, \cmt{communicates} with ground services to process it, and \cmt{dispatches} qualified personnel to \cmt{assist distressed individuals.}. This work considers \cmt{different degradation sources} as reported in the official documentation \cite{galileoperformances,galileoossdd,galileoosperformancereport}. The satellite can be in one of three states: nominal, degraded, or severely degraded. Specific anomalies, such as loss of communication with the ground station, cause each degradation status with elapsed time in such degradation. These anomalies are collected through availability monitoring as described in \cite{galileoperformances,galileoossdd,galileoosperformancereport}. The system model is specified as a Continuous-Time Markov Chain (CTMC) \cite{Kwiatkowska2007}. Assuming constant failure and repair rates for the SAR communication services, the time to failure and repair are modeled as exponentially distributed random variables. \cmt{Furthermore}, this work introduces a \cmt{new} approach by integrating human factors into the model. Unlike \cmt{earlier} models primarily focused on technological aspects, this research \cmt{integrates} human behavior as an \quot{sensor} within the system \cite{nunes2018practical}. \cmt{This analysis investigates the crucial role of human factors, such as psychological states and environmental conditions, in the timely transmission of rescue signals. It highlights the significant impact of human behavior—particularly under stress—on the effectiveness of search and rescue (SAR) operations. By conceding these dynamics, we can enhance the overall efficiency and success of SAR missions.}

\paragraph*{Outline}The remainder of this paper is structured as follows. Section~\ref{works} reviews related work. Section~\ref{preliminaries} provides a brief background on CTMCs and the PRISM language. Section~\ref{sattelitemodel} presents our proposed modeling approach for SAR systems. We then perform a quantitative analysis of the SAR services scenarios in Section~\ref{usecase}. Finally, Section~\ref{conclusion} concludes the paper and suggests directions for future research.