This paper presents an approach based on CTMCs to model the communication services of the SAR/Galileo system. The captured model incorporates multiple degradation scenarios related to the observed and monitored communication between satellite systems and ground stations. We leverage the PRISM model checker for quantitative analysis, considering availability parameters and the evolving status of the distressed person.


An evaluation assessed which degradation source contributes most significantly to system failures and reduced reliability. Multiple factors were investigated, including loss of communication with monitoring ground stations and monitored failures attributed to human causes or environmental factors (the documentation does not explicitly mention the accurate sources of failures). The results demonstrate that ground stations responsible for monitoring signals are the most active sources of failures. Also, an evaluation has been performed to assess the human's capability to react to the crisis in conjunction with the system status. The results demonstrate that the \cmt{system and human parameters significantly influence performance}. 

Future work will consider incorporating additional parameters, such as the number of workstations, and exploring the implications of other formalisms, such as stochastic games, for assessing the reliability of human behavior in distress situations. The extension will also involve a comparative analysis between statistical model checkers and probabilistic model checkers to investigate the size of the resulting models and the feasibility of model verification.